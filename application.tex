Now that we stated the theorem \ref{thm:main}, we give examples of a realizability models satisfying the conditions (and thus, realizing $\FT$).

The first example is the one given before. For each $\enviro \in \Oracle$, we can define the EFdata $\EFSATE$ from \Cref{rmk:EFSigma}. For $\enviro, \enviroT \in \Oracle$ such that $\enviro\infOr \enviroT$, we have a morphism of EFdata given by inclusion. For each $\enviro$, the EFdata $\EFSATE$ has continuity, explicit existential and unbounded search, hence the fact that it satisfies $\FT$.

The second example is the historical one: the second Kleene algebra $K_2$ \cite{Kleene1965} satisfies the premisses. The preorder is the trivial one, with only one element. Continuity is by definition of the PCA (it is the set of continuous functions $(\bN \to \bN) \to \bN$), unbounded search can be defined for any PCA by encoding the combinator $Y$ in combinatory logic. Any function $\bN \to \bN$ has a code, hence the condition that a path exists for any decidable tree encoded. Finally, explicit existential is true for intuitionnistic PCAs. We thus find back the usual result that $K_2$ satisfies $\FT$.

A third example is the model we based our generalization on: the model by Lubarsky in \cite{LubRat13}.

The existence of a path in a possible future is essential. For example, taking the realizability model given by $\SATE$, with local computation at some context $\enviro\in\Oracle$, the model does not realize $\FT$. To show this, we use the Kleene tree relativized to $\enviro$-computable functions. For any $\enviro$-computable path, given by a code $e$, we can simulate $e$ for increasing time duration until the finite path gets out of the tree, which will always happen because the infinite path in the Kleene tree are all uncomputable. This thus gives a bar for the Kleene tree, but this bar can't be uniform as the length of lists in this tree is unbounded.

Finally, we advocate that the restriction of working on EFdatas instead of just EF is not a limitation in practice. The additionnal data, indeed, always exist inside the realizability topos given by an EF.

Take an EF $(E,\Phi,\relEFdot)$. Using the UFam construction to get a tripos, and then using the tripos to topos construction, we can describe the objects of this topos. We focus on one of its full subcategories, given by the assemblies which are the following data:
\begin{itemize}
\item a set $X$
\item a function $E_X : X \to \Phi$
\end{itemize}
and morphisms $(X,E_X) \to (Y,E_Y)$ are given by
\begin{itemize}
\item a function $f : X \to Y$
\item an evidence $\relEF{e}{E_X(x)}{E_Y(f(x))}$ uniform on $x \in X$
\end{itemize}
This exactly corresponds to the way we define codes for system T types, so every construction in the EFdata can be translated inside the realizability topos. The object associated to $(\bN,\cod -)$ in the realizability topos is a natural number object (NNO) \cite{Lawvere1963}: $e_0$ defines the element $0 \in \bN$, $e_S$ defines the function $\bN \to \bN$ and $e_{\rec}$ ensures the initiality of $1+\bN \xrightarrow{[0,S]} \bN$ among the algebras for $X\mapsto 1+X$.

%\tlnote{Ajouter une ref pour le NNO ? Ca vient de Lawvere 1963 apparemment, mais je sais pas si c'est pertinent de mettre une citation pour ça.}

Hence, to get an EFdata from an EF $\EFE$, one needs to find a NNO inside the realizability topos of $\EFE$, for which the initial algebra property is assured in a uniform way (\textit{i.e.} by the same witness $e \in E$), and pull back the associated data to make $e_0,e_S,e_{\rec}$. In practice, any realizability model of interest possesses such an object, as logically relevant models need arithmetical language (and thus a NNO) and it is natural to keep the same constructor for every recursive definition.

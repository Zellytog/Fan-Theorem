\subsection{Lambda-calculus}

The proof we adapt from \cite{LubRat13} uses the Turing-machine realizability relation over a Kripke model of IZF. This Kripke model is made on the frame $\omega^{<\omega}$ and every world in this frame is equiped with a finite family of oracles: the realizers then use only these oracles for their computation.

Simplifying this construction, we consider a lambda-calculus which, similarly to the realizers of the model mentionned above, is able to call a finite number of oracles. To implement the call to oracles primitive, we add a family of terms $\oracle_n$ representing the $n$-th oracle of our family.

\begin{definition}[Lambda-terms]
  We fix an enumerable set $\Xlam$ of $\lambda$-variables which will be written
  $x,y,\ldots$ and define inductively the set $\Lambda$ of lambda-terms by:
  \begin{align*}
    t,u  &\Coloneq  x \mid \lambda x.t \mid t\;u \\
    & \mid \langle t,u\rangle \mid \pi_1\;t \mid \pi_2\;t \\
    & \mid \lamZ \mid \lamS\;t \mid \rec_{\Nat}\;t\;u\;v\\
    & \mid \rtt \mid \rff \mid \rec_{\Bool}\;t\;u\;v\\
    & \mid \nil \mid t\append u \mid \rec_{\List}\;t\;u\;v\\
    & \mid \oracle_n
  \end{align*}
  where $n$ ranges over $\bN$.
\end{definition}

We assume the usual convention of $\alpha$-renaming of bound variables.

Using our multiple worlds analogy, we parametrize the reduction on terms by a list of functions $\bN \to \bB$. In place of the frame $\omega^{<\omega}$ previously mentionned, our set of worlds is $(\bN \to \bB)^{<\omega}$, but every definition can be restricted to a subset of $\bN \to \bB$, for example by taking only functions of some given degree. The model of \cite{LubRat13} is the case where we restrict the $i$-th element of the list to be in a degree $\degreeD_i$, where $(\degreeD_i)$ is a fixed countable family of Turing degrees.

\begin{definition}[Reduction context]
  Let $\funB$ be the set of set-functions $\bN \to \bB$. We denote by $\Oracle$ the set of lists of set-functions $\bN \to \bB$:
  \[\Oracle \defeq \List(\funB)\]
  Elements of this set will be called reduction contexts. For any $\enviro \in \Oracle$, the $i$-th function of the list will be written $\enviro[i]$ and the length of $\enviro$ will be written $|\enviro|$.
\end{definition}

We assume the usual convention for substitution without capture of free variables.

\begin{notation}
  Given a partial map $\rho : \Xlam \partialto \Lambda$ and a term $t$, the simultaneous substitution of $t$ by $\rho$ is denoted $\rho(t)$. If $\rho$ is the function defined on the only point $x \in \Xlam$ by $x \mapsto u$, then we will write $t[u/x]$.
\end{notation}

The reduction itself is the expected $\beta$-reduction, enriched by the reduction, for
any set-function $f : \bN \to \bB$ in the $i$-th place of the reduction context:
\[\oracle_i \;\encode{n} \mapsto \encode{f(n)}\]
where $\encode{n}$ means $\lamS^n\;\lamZ$, the canonical encoding of integers in this lambda-calculus, and the encoding of booleans is $\rtt$ and $\rff$, corresponding respectively to $1 \in \bB$ and $0 \in \bB$.

\begin{definition}[Reduction]
  Let $\oracle \in \Oracle$.
  We define the relation $\mapsto_\sigma$ of immediate reduction by the following rules:
  \[\begin{array}{lclr}
    (\lambda x.t)u & \rediE & t[u/x] \\
    \pi_i \langle t_1,t_2\rangle & \rediE & t_i & \forall i \in \{1,2\} \\
    \rec_{\Nat}\;t\;u\;\lamZ & \rediE & t \\
    \rec_{\Nat}\;t\;u\;(\lamS\;v) & \rediE & u\;v\;(\rec_{\Nat}\;t\;u\;v) \\
    \rec_{\Bool}\;t\;u\;\rtt & \rediE & t \\
    \rec_{\Bool}\;t\;u\;\rff & \rediE &u \\
    \rec_{\List}\;t\;u\;\nil & \rediE &t \\
    \rec_{\List}\;t\;u\;(v \append w) & \rediE &u\;v\;w\;(\rec_{\List}\;t\;u\;w) \\
    \oracle_i\;\encode{n} & \rediE & \encode{\enviro[i](n)} & \forall i < |\enviro|
  \end{array}\]

  The reduction relation $\redE$ is the smallest relation containing $\rediE$ and which is compatible, meaning that it is closed by subterms. We write $\redERT$ for the reflexive and transitive closure of $\redE$.
\end{definition}

%It is folklore to check that $\redE$ is confluent.

As the reduction depends on the reduction context, we show that extending the reduction context also extends the reduction.

\begin{notation}
  For $\enviro,\enviroT \in \Oracle$, we write $\enviro \infOr \enviroT$ when $\enviro$ is a prefix of $\enviroT$.

  %We also write the cylinder on $\enviro$ as follows:
  %\[\cyl\enviro \defeq \{\enviroT \in \Oracle \mid \enviro \infOr \enviroT\}\]
\end{notation}

\begin{proposition}
  If $\enviro, \enviroT \in \Oracle$ are such that $\enviro \infOr \enviroT$, then $\redE \subseteq \red_{\enviroT}$.
\end{proposition}

\begin{proof}
  It suffices to show that $\rediE\subseteq \redi_{\enviroT}$, but this is straightforward by just comparing the list of redexes.
\end{proof}

The standard interpretation of $\Prop$ in realizability is the set $\SAT$ of saturated sets. In our case, as saturation is a property about (anti)reduction, we parametrize the notion of saturation by a reduction context.

The notion of truth value in our model is thus, in a world $\enviro$, the set of saturated sets for $\rediE$. This truth value is local, in the sense that it only speaks about $\enviro$, so the adapted notion of truth value (for the Kripke interpretation) is given by family of saturated sets, for each $\enviro$.

\begin{definition}[Saturated set]
  Let $\enviro\in\Oracle$ and $A \subseteq \Lambda$.
  We call $A$ a $\enviro$-saturated set if it is closed by anti-reduction for $\redE$, and write $\SATE$ the set of saturated sets:
  \[A \in \SATE \defeq \forall t,u \in \Lambda, t \rediE u \implies u \in A \implies t \in A\]
  We define the set of global saturated subsets as follows:
  \[\SAT \defeq \left\{\left.(A_{\enviro}) \in \prod_{\enviro\in\Oracle} \SATE \right\vert
  \forall \enviro,\enviroT\in\Oracle, \enviro \infOr \enviroT \implies A_{\enviro} \subseteq A_{\enviroT}\right\}\]
\end{definition}

As for usual saturated set, the different notions of saturated sets are complete lattice.

\begin{proposition}
  For any $\enviro \in \Oracle$, $\SATE$ is a complete sub-lattice of $\powerset(\Lambda)$. $\SAT$ is also a complete lattice for the pointwise inclusion and with the pointwise union and intersection.
\end{proposition}

%We now look at the relation between inclusion of reduction contexts and reductions.

%\begin{proposition}
%  For $\enviro\infOr\enviroT$, the following inclusion holds:
%  \[\SATP{\enviroT}\subseteq \SATP{\enviro}\]
%\end{proposition}

%\begin{proof}
%  Let $\enviro,\enviroT \in \Oracle$ such that $\enviro\infOr\enviroT$. Let $A \in \SATP{\enviroT}$, $t,u \in \Lambda$ such that $\forall \enviroTT \in \cyl{\enviro}, t \redP{\enviroTT} u$ and $u \in A$. Then, we know that for any $\enviroTT \in \cyl{\enviroT}, t \redP{\enviroTT} u$ as $\cyl{\enviroT}\subseteq \cyl{\enviro}$, so $t \in A$ as $A$ is $\enviroT$-saturated.
%\end{proof}

%We also give a lemma which simplfies some proofs of saturation.

%\begin{lemma}\label{lem:inter_SAT}
%  Let $(A_\enviro)_{\enviro\in\Oracle}$ be a family of sets such that $\forall \enviro\in\Oracle, A_\enviro \in \SATE$. Then $\displaystyle\bigcap_{\enviro \in \Oracle} A_\enviro \in \SAT$.
%\end{lemma}

%\begin{proof}
%  Let $t,u \in \Lambda$ such that
%  \[\forall \enviro \in \Oracle, t \redE u\]
%  and $u \in \displaystyle\bigcap_{\enviro \in \Oracle} A_\enviro$. Then for any $\enviro \in \Oracle$, $u \in A_\enviro$ so (as $A_\enviro$ is saturated) $t \in A_\enviro$. So $t \in \displaystyle\bigcap_{\enviro\in\Oracle} A_\enviro$.
%\end{proof}

The fact that each $\SATE$ and $\SAT$ are taken to be truth value means that we will use them to interpret $\Prop$ in the semantic interpretation of our HOL syntax. We thus show that those sets can be equiped with a Heyting pre-algebra structure.

The structure of Heyting pre-algebra on $\SATE$ is the expected one: $A \leq B$ means that there is a uniform mapping of elements of $A$ to elements of $B$, and the set of those mapping (is saturated and) constitutes the set $A \implies B$. For the structure of $\SAT$, however, the structure is defined in a more subtle way. $\varA\leq \varB$ means that the mapping witnessing the inequality is also uniform on $\enviro$. The implication $\implies$ in $\SAT$ is then defined in a Kripke-style way, as the elements inside a world $\enviro$ also have to work for any future world $\enviroT\supOr \enviro$. This definition of $\implies$ is the reason why we impose monotonicity in the family $\varA\in \SAT$.

\begin{proposition}
  Let $\enviro \in \Oracle$.
  Let $A,B \in \SATE$ and $\varA,\varB \in \SAT$. Then let
  \begin{align*}
    A \infSATE B &\defeq \exists t \in \Lambda, \forall u \in A, tu \in B\\
    A \impliesSATE B &\defeq \{t \in \Lambda \mid \forall u \in A, tu \in B\}\\
    A \landSATE B &\defeq \{t \in \Lambda\mid (\pi_1\;t \in A)\land (\pi_2\;t \in B)\}\\
    \varA \infSAT \varB &\defeq \exists t \in \Lambda, \forall \enviro \in \Oracle, \forall u \in \termAE, tu \in \termBE\\
    \varA \impliesSAT \varB &\defeq \enviro \mapsto \{ t \in \Lambda \mid \forall \enviroT \supOr \enviro, \forall u \in \termA{\enviroT}, tu \in \termB{\enviroT}\}\\
    \varA\landSAT\varB &\defeq \enviro \mapsto \{ t \in \Lambda \mid (\pi_1\;t\in \termAE) \land (\pi_2\;t \in \termBE)\}
  \end{align*}
  This defines a Heyting pre-algebra.
\end{proposition}

\begin{proof}
  The fact that the two operations are well defined is standard and uses the compatibility of the relation.
  
  Let's show that $\infSATE$ is the right adjoint to $\landSATE$. This means that for any $A,B,C \in \SATE$:
  \[A \infSATE B \impliesSATE C \iff A \landSATE B \infSATE C\]
  \begin{itemize}
  \item suppose there is $t$ such that for any $u \in A$, $tu \in B \impliesSATE C$. Thus, the function $(\lambda x. t\;(\pi_1\;x)\;(\pi_2\;x))$ is a witness that $A \landSATE B \infSATE C$
  \item conversely, if there is $t$ such that for any $u \in A \landSATE B$, $tu \in C$, then the function $\lambda x.\lambda y.t \langle x,y\rangle$ is a witness that $A \infSATE B \impliesSATE C$.
  \end{itemize}

  We do the same for $\SAT$:
  \[\varA \infSAT \varB \impliesSAT \varC \iff \varA\landSAT \infSAT \varC\]
  \begin{itemize}
  \item suppose there is $t$ such that for any $\enviro,\enviroT\in\Oracle$, $u \in \termAE, v \in \termB{\enviroT}$ and such that $\enviro \infOr \enviroT$, $t\;u\;v \in \termC{\enviroT}$. Then, for $\enviro \in \Oracle$ and $u \in \termAE\landSAT\termBE$, $\pi_1\;u\in\termAE$ and $\pi_2\;u\in\termBE$, so $t\;(\pi_1\;u)\;(\pi_2\;u)\in\termCE$.
  \item suppose there is $t$ such that for any $\enviro\in\Oracle$ and $u \in \termAE \landSAT \termBE$, $tu \in \termCE$. Then, let $\enviro,\enviroT\in\Oracle$ such that $\enviro\infOr \enviroT$ and $u \in \termAE$. By inclusion, $u \in \termA{\enviroT}$, so for any $v \in \termB{\enviroT}$, $t\;\langle u,v\rangle \in \termC{\enviroT}$, so by antireduction $\lambda x\;y.t\;\langle x,y\rangle$ is a witness that $\varA\infSAT \varB \impliesSAT \varC$.
  \end{itemize}

  Hence $\SATE$ and $\SAT$ are Heyting pre-algebras.
\end{proof}

\begin{remark}
  The pre-order $\infSATE$ contains $\subseteq$ with the witness $t = \lambda x.x$. For this reason, $\bigcap$ defines a lower bound, even though not the greatest.
\end{remark}

%Finally, we give the Heyting pre-algebra we will use for our construction. This one is given by in a similar way as for $\SAT$, but we require a Kripke-like condition for the implication.

%\begin{definition}[$\SAT$ family Heyting pre-algebra]
%  We define the following Heyting pre-algebra. For the carrier set, let
%  \[\HeytingFamily \defeq \left\{\left. (A_\enviro)_{\enviro \in \Oracle} \in \prod_{\enviro \in \Oracle} \SATE \right\vert \forall \enviroT \in \Oracle, \enviro \infOr \enviroT \implies A_\enviro \subseteq A_{\enviroT}\right\}\]
%  We define the pre-order as
%  \[\varA \InfSATFam \varB \defeq \exists t \in \Lambda, \forall \enviro \in \Oracle, \forall u \in \termAE, t\;u \in \termBE\]
%  the conjunction and the implication respectively as
%  \begin{align*}
%    \varA \AndSATFam \varB &\defeq \enviro \longmapsto \{t \in \Lambda \mid (\pi_1\;t\in \termAE)\land (\pi_2\;t \in \termBE)\} \\
%    \qquad \varA \ImplSATFam \varB &\defeq \enviro \longmapsto \{ t \in \Lambda \mid \forall \enviroT \in \cyl{\enviro}, \forall u \in \termA{\enviroT}, t\;u \in \termB{\enviroT}\}
%  \end{align*}
%\end{definition}

%\begin{proof}
%  We show that this is indeed a Heyting pre-algebra with arbitrary meet and join. First, for the meet and join, the fact that each $\SATE$ is stable by arbitrary intersection and union means that we can define
%  \[\bigcap_{i \in I} (A_{\enviro,i})_{\enviro \in \Oracle} \defeq \left(\bigcap_{i \in I} A_{\enviro,i}\right)_{\enviro \in \Oracle}\]

%  We are left to show the adjunction between $\AndSATFam$ and $\ImplSATFam$. Let $\varA$, $\varB$ and $\varC$ be families, then stating that $(\varA \AndSATFam \varB) \InfSATFam \varC$ and $\varA \InfSATFam (\varB \ImplSATFam \varC)$ mean respectively:
%  \begin{itemize}
%  \item we find a term $t$ such that if $\pi_1 u \in \termAE$ and $\pi_2 u \in \termBE$ for any $\enviro \in \Oracle$, then $t\;u \in \termCE$ for any $\enviro \in \Oracle$
%  \item we find a term $t$ such that, for any $\enviro,\enviroT \in \Oracle$ such that $\enviro\infOr \enviroT$, $u \in \termAE, v \in \termB{\enviroT}$, we have $t\;u\;v \in \termC{\enviroT}$
%  \end{itemize}
%  using the monotonicity of the families, the two terms can be converted one into the other respectively by the two following functions:
%  \[t \longmapsto \lambda u.\lambda v.t\;\langle u,v\rangle \qquad
%  t \longmapsto \lambda u. t\;(\pi_1\;u)\;(\pi_2\;u)\]

%  Hence this gives us a Heyting pre-algebra with a meet and join operation.
%\end{proof}

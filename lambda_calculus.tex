\subsection{Lambda-calculus}

The lambda-calculu we use is a variant of the usual lambda-calculus, enriched with
constructors and recursors for integers, booleans and lists of objects. We also add
pairs and a family of oracles.

\begin{definition}[Lambda-terms]
  We fix an enumerable set $\Xlam$ of $\lambda$-variables which will be written
  $x,y,\ldots$ and define inductively the set $\Lambda$ of lambda-terms by:
  \begin{align*}
    t,u  &\Coloneq  x \mid \lambda x.t \mid t\;u \\
    & \mid \langle t,u\rangle \mid \pi_1\;t \mid \pi_2\;t \\
    & \mid \lamZ \mid \lamS\;t \mid \rec_{\Nat}\;t\;u\;v\\
    & \mid \rtt \mid \rff \mid \rec_{\Bool}\;t\;u\;v\\
    & \mid \nil \mid t\append u \mid \rec_{\List}\;t\;u\;v\\
    & \mid \oracle_n
  \end{align*}
  where $n$ ranges over $\bN$.
\end{definition}

We assume the usual convention of $\alpha$-renaming of bound variables.

As the family $(\oracle_n)_{n \in \bN}$ represents oracles, the reduction will be parametrised by a list of functions $\bN \to \bB$.

\begin{definition}[Reduction context]
  Let $\funB$ be the set of set-functions $\bN \to \bB$. We denote by $\Oracle$ the set of lists of set-functions $\bN \to \bB$:
  \[\Oracle \defeq \List(\funB)\]
  Elements of this set will be called reduction contexts. For any $\enviro \in \Oracle$, the $i$-th function of the list will be written $\enviro[i]$ and the length of $\enviro$ will be written $|\enviro|$.
\end{definition}

We assume the usual convention for substitution without capture of free variables.

\begin{notation}
  Given a partial map $\rho : \Xlam \partialto \Lambda$ and a term $t$, the simultaneous substitution of $t$ by $\rho$ is denoted $\rho(t)$. If $\rho$ is the function defined on the only point $x \in \Xlam$ by $x \mapsto u$, then we will write $t[u/x]$.
\end{notation}

The reduction itself is the expected $\beta$-reduction, enriched by the reduction, for
any set-function $f : \bN \to \bB$ in the $i$-th place of the reduction context:
\[\oracle_i \;\encode{n} \mapsto \encode{f(n)}\]
where $\encode{n}$ means $\lamS^n\;\lamZ$, the canonical encoding of integers in this lambda-calculus, and the encoding of booleans is obviously $\rtt$ and $\rff$, corresponding respectively to $1 \in \bB$ and $0 \in \bB$.

\begin{definition}[Reduction]
  Let $\oracle \in \Oracle$.
  We define the relation $\mapsto_\sigma$ of immediate reduction by the following rules:
  \[\begin{array}{lclr}
    (\lambda x.t)u & \rediE & t[u/x] \\
    \pi_i \langle t_1,t_2\rangle & \rediE & t_i & \forall i \in \{1,2\} \\
    \rec_{\Nat}\;t\;u\;\lamZ & \rediE & t \\
    \rec_{\Nat}\;t\;u\;(\lamS\;v) & \rediE & u\;v\;(\rec_{\Nat}\;t\;u\;v) \\
    \rec_{\Bool}\;t\;u\;\rtt & \rediE & t \\
    \rec_{\Bool}\;t\;u\;\rff & \rediE &u \\
    \rec_{\List}\;t\;u\;\nil & \rediE &t \\
    \rec_{\List}\;t\;u\;(v \append w) & \rediE &u\;v\;w\;(\rec_{\List}\;t\;u\;w) \\
    \oracle_i\;\encode{n} & \rediE & \encode{\enviro[i](n)} & \forall i < |\enviro|
  \end{array}\]

  The reduction relation $\redE$ is the smallest relation containing $\rediE$ and which is compatible, meaning that it is closed by subterms. We write $\redERT$ for the reflexive and transitive closure of $\redE$.
\end{definition}

It is folklore to check that $\redE$ is confluent.

\begin{notation}
  For $\enviro,\enviroT \in \Oracle$, we write $\enviro \infOr \enviroT$ when $\enviro$ is a prefix of $\enviroT$. We also write the cylinder on $\enviro$ as follows:
  \[\cyl\enviro \defeq \{\enviroT \in \Oracle \mid \enviro \infOr \enviroT\}\]
\end{notation}

\begin{proposition}
  If $\enviro, \enviroT \in \Oracle$ are such that $\enviro \infOr \enviroT$, then $\redE \subseteq \red_{\enviroT}$.
\end{proposition}

\begin{proof}
  It suffices to show that $\rediE\subseteq \redi_{\enviroT}$, but this is straightforward by just comparing the list of redexes.
\end{proof}

The truth values of our predicates will be subsets of $\Lambda$. As is standard in realizability, thoses subsets need to be closed by anti-reduction.

\begin{definition}[Saturated set]
  Let $\enviro\in\Oracle$ and $A \subseteq \Lambda$.
  We call $A$ a $\enviro$-saturated set if it is closed by anti-reduction for $\redE$, and write $\SATE$ the set of saturated sets:
  \[A \in \SATE \defeq \forall t,u \in \Lambda, (\forall \enviroT \in \cyl{\enviro}, t \redP{\enviroT} u) \implies u \in A \implies t \in A\]
  We define the set of saturated subsets as follows:
  \[A \in \SAT \defeq \forall t,u \in \Lambda, (\forall \enviro \in \Oracle, t \redE u) \implies u \in A \implies t \in A\]
\end{definition}

\begin{remark}
  Using the fact that $\redE\subseteq\red_{\enviroT}$ for $\enviro\infOr\enviroT$, we know that if $t \redE u$ and $u \in A$, for $A \in \SATE$, then $t \in A$. Thus the condition of being saturated is a strengthening of the naive definition, which would be
  \[\forall t,u \in \Lambda, t \redE u \implies u \in A \implies t \in A\]
\end{remark}

The following result is folklore.

\begin{proposition}
  For any $\enviro \in \Oracle$, $\SATE$ is a complete sub-lattice of $\powerset(\Lambda)$. $\SAT$ is also a complete sub-lattice of $\powerset(\Lambda)$.
\end{proposition}

We now look at the relation between inclusion of reduction contexts and reductions.

\begin{proposition}
  For $\enviro\infOr\enviroT$, the following inclusion holds:
  \[\SATP{\enviroT}\subseteq \SATP{\enviro}\]
\end{proposition}

\begin{proof}
  Let $\enviro,\enviroT \in \Oracle$ such that $\enviro\infOr\enviroT$. Let $A \in \SATP{\enviroT}$, $t,u \in \Lambda$ such that $\forall \enviroTT \in \cyl{\enviro}, t \redP{\enviroTT} u$ and $u \in A$. Then, we know that for any $\enviroTT \in \cyl{\enviroT}, t \redP{\enviroTT} u$ as $\cyl{\enviroT}\subseteq \cyl{\enviro}$, so $t \in A$ as $A$ is $\enviroT$-saturated.
\end{proof}

We also give a lemma which simplfies some proofs of saturation.

\begin{lemma}\label{lem:inter_SAT}
  Let $(A_\enviro)_{\enviro\in\Oracle}$ be a family of sets such that $\forall \enviro\in\Oracle, A_\enviro \in \SATE$. Then $\displaystyle\bigcap_{\enviro \in \Oracle} A_\enviro \in \SAT$.
\end{lemma}

\begin{proof}
  Let $t,u \in \Lambda$ such that
  \[\forall \enviro \in \Oracle, t \redE u\]
  and $u \in \displaystyle\bigcap_{\enviro \in \Oracle} A_\enviro$. Then for any $\enviro \in \Oracle$, $u \in A_\enviro$ so (as $A_\enviro$ is saturated) $t \in A_\enviro$. So $t \in \displaystyle\bigcap_{\enviro\in\Oracle} A_\enviro$.
\end{proof}

The $\SATE$ (and $\SAT$) are also Heyting algebras.

\begin{proposition}
  Let $\sigma \in \Oracle$.
  Let $A,B \in \SATE$ (resp. $\SAT$). Then let
  \begin{align*}
    A \infSATE B &\defeq \exists t \in \Lambda, \forall u \in A, tu \in B\\
    A \impliesSATE B &\defeq \{t \in \Lambda \mid \forall u \in A, tu \in B\}\\
    A \landSATE B &\defeq \{t \in \Lambda\mid (\pi_1\;t \in A)\land (\pi_2\;t \in B)\}
  \end{align*}
  This defines a Heyting pre-algebra.
\end{proposition}

\begin{proof}
  We only prove the result for $\SATE$, the reasonning is the same for $\SAT$.
  
  The fact that the two operations are well defined is standard and uses the compatibility of the relation.
  
  Let's show that $\infSATE$ is the right adjoint to $\landSATE$. This means that for any $A,B,C \in \SATE$:
  \[A \infSATE B \impliesSATE C \iff A \landSATE B \infSATE C\]
  \begin{itemize}
  \item suppose there is $t$ such that for any $u \in A$, $tu \in B \impliesSATE C$. Thus, the function $(\lambda x. t\;(\pi_1\;x)\;(\pi_2\;x))$ is a witness that $A \landSATE B \infSATE C$
  \item conversely, if there is $t$ such that for any $u \in A \landSATE B$, $tu \in C$, then the function $\lambda x.\lambda y.t \langle x,y\rangle$ is a witness that $A \infSATE B \impliesSATE C$.
  \end{itemize}

  Hence $\SATE$ is a Heyting pre-algebra.
\end{proof}

Finally, we give the Heyting pre-algebra we will use for our construction. This one is given by in a similar way as for $\SAT$, but we require a Kripke-like condition for the implication.

\begin{definition}[$\SAT$ family Heyting pre-algebra]
  We define the following Heyting pre-algebra. For the carrier set, let
  \[\HeytingFamily \defeq \left\{\left. (A_\enviro)_{\enviro \in \Oracle} \in \prod_{\enviro \in \Oracle} \SATE \right\vert \forall \enviroT \in \Oracle, \enviro \infOr \enviroT \implies A_\enviro \subseteq A_{\enviroT}\right\}\]
  We define the pre-order as
  \[\varA \InfSATFam \varB \defeq \exists t \in \Lambda, \forall \enviro \in \Oracle, \forall u \in \termAE, t\;u \in \termBE\]
  the conjunction and the implication respectively as
  \begin{align*}
    \varA \AndSATFam \varB &\defeq \enviro \longmapsto \{t \in \Lambda \mid (\pi_1\;t\in \termAE)\land (\pi_2\;t \in \termBE)\} \\
    \qquad \varA \ImplSATFam \varB &\defeq \enviro \longmapsto \{ t \in \Lambda \mid \forall \enviroT \in \cyl{\enviro}, \forall u \in \termA{\enviroT}, t\;u \in \termB{\enviroT}\}
  \end{align*}
\end{definition}

\begin{proof}
  We show that this is indeed a Heyting pre-algebra with arbitrary meet and join. First, for the meet and join, the fact that each $\SATE$ is stable by arbitrary intersection and union means that we can define
  \[\bigcap_{i \in I} (A_{\enviro,i})_{\enviro \in \Oracle} \defeq \left(\bigcap_{i \in I} A_{\enviro,i}\right)_{\enviro \in \Oracle}\]

  We are left to show the adjunction between $\AndSATFam$ and $\ImplSATFam$. Let $\varA$, $\varB$ and $\varC$ be families, then stating that $(\varA \AndSATFam \varB) \InfSATFam \varC$ and $\varA \InfSATFam (\varB \ImplSATFam \varC)$ mean respectively:
  \begin{itemize}
  \item we find a term $t$ such that if $\pi_1 u \in \termAE$ and $\pi_2 u \in \termBE$ for any $\enviro \in \Oracle$, then $t\;u \in \termCE$ for any $\enviro \in \Oracle$
  \item we find a term $t$ such that, for any $\enviro,\enviroT \in \Oracle$ such that $\enviro\infOr \enviroT$, $u \in \termAE, v \in \termB{\enviroT}$, we have $t\;u\;v \in \termC{\enviroT}$
  \end{itemize}
  using the monotonicity of the families, the two terms can be converted one into the other respectively by the two following functions:
  \[t \longmapsto \lambda u.\lambda v.t\;\langle u,v\rangle \qquad
  t \longmapsto \lambda u. t\;(\pi_1\;u)\;(\pi_2\;u)\]

  Hence this gives us a Heyting pre-algebra with a meet and join operation.
\end{proof}

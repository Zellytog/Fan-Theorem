Friedman-Simpson’s original program of reverse mathematics,
as is also the case for most of standard mathematics,
has been developed in classical subsystems of second-order
arithmetic. As such, (classical) reverse mathematics
presents various limitations from a constructive point of view,
since for instance they are unable do distinguish between
a statement and its contrapositive (e.g. dependent choice and
the bar induction principles).
The case of (Weak) Kőnig Lemma (WKL) and Fan Theorem (FT)
is particularly interesting in that regard:
while KL is well-known to implies FT, and if constructivists like
Brouwer rejected the former while admitting the latter,
the converse implication has not been much studied for years.
It is only recently that a growing enthusiasm for constructive reverse
mathematics pushed towards a finer-grained analysis of the connection
between such principles.
In addition to intuitionnistic reverse mathematics, the realizability approach
to logical principles adds a computational meaning to purely logical statements.
We follow this approach by giving a computational meaning to Brouwer's Fan
Theorem: building on recent work by Lubarsky and Rathjen, we construct a
realizability interpration of higher-order logic validating FT while refuting WKL. This interpretation relies on a $\lambda$-calculus extended with oracles while preserving a notion of continuity for realizers. Last, we show the robustness of this approach by identifying, in the abstract and general setting of evidenced frames, sufficient computational conditions entailing FT.

%Building on recent work by Lubarsky and Rathjen,
%we give here a computational perspective on the separation between
%FT and WKL, by first providing a realizability interpretation of second-order
%logic based on the $\lambda$-calculus that validates Brouwer's Fan Theorem
%while negating Weak König Lemma.

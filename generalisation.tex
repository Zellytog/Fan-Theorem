\subsection{Generalizing the argument}

We will now generalize our proof of $\FT$. The main ideas we used are:
\begin{itemize}
\item the realizability model allows one to define at least system T functions (encodings allow one to talk about functions $\List(\Nat) \to \Bool$ for example)
\item a function $f$ in the model can be called on a function $g$ defined later, $g$ can be non computable at the stage where $f$ is defined.
\item the functions $(\Nat \to \Nat) \to \Nat$ are continuous, in the sense that for any $f : (\Nat \to \Nat) \to \Nat$ and $\alpha : \Nat \to \Nat$ with a code, there exists a modulus $n$ such that $\forall \beta, \restr\alpha n = \restr \beta n \implies f(\alpha) = f(\beta)$.
\item the computationnal model can do an unbounded search with only the information in the meta-theory that the search will terminate.
\end{itemize}

The Kripke semantic styled realizability relation we gave, and the fact we state a property about a possible future, strongly leads us to a generalization on some Kripke model. The usual way of giving a Kripke model is to consider a category $\bC$, a pre-ordered set $(W, \leq)$ and to give a functor $F : W \to \bC$. We will follow this by stating exactly the informations we wish to preserve in our morphisms.

This leads us to the definition of evidenced frame with data types.

\begin{definition}[EF with data types]
  An evidenced frame with data types (EFdata, for short), is a tuple $(E,\Phi,\relEFdot, \cod \cdot, e_0, e_S, e_{\rec})$ with:
  \begin{itemize}
  \item $(E,\Phi,\relEFdot)$ is an evidenced frame.
  \item $\cod \cdot : \bN \to \Phi$ is an encoding function. We extend it naturally for types on
    \[T,U \Coloneq \bN \mid T \to U\]
    by, for any $f : T \to U$:
    \[\cod f \defeq \prod_{x \in T} \PropImpl{\cod x}{\cod{f(x)}}\]
  \item $\relEF{e_0}{\top}{\cod 0}$
  \item for all $n \in \bN$, $\relEF{e_S}{\cod n}{\cod{n + 1}}$
  \item for all type $T$ as previously defined, let $\rec_T$ be the recursor $T \to (\bN \to T \to T) \to \bN \to T$, then $\relEF{e_{\rec}}{\top}{\cod{\rec_T}}$
  \end{itemize}
\end{definition}

\begin{remark}
  For any $n \in \bN$, $\cod n$ is realized by $S^n 0$. Also, any system T definable function is realized by the naturally associated term.
\end{remark}

Now, we define continuity, unbounded search and explicit existential.

\begin{definition}[Continuity]
  Let $(E,\Phi,\relEFdot, \cod\cdot, e_0,e_S,e_{\rec})$ be an EFdata. This EFdata is said to have continuity if
  \begin{multline*}
    \forall f : (\bN \to \bN) \to \bN, \forall \alpha : \bN \to \bN, \forall \relEF{t}{\top}{\cod f},
    \\\forall \relEF{u}{\top}{\cod \alpha}, \exists n \in \bN,
    \forall \beta:\bN\to\bN,\restr \alpha n = \restr \beta n \implies f(\alpha) = f(\beta)
  \end{multline*}
\end{definition}

\begin{definition}[Unbounded search]
  An EFdata is said to have unbounded search if there exists $e$ such that, for any $f : \bN \to \bN$ having at least one zero,
  \[\relEF{e}{\cod f}{(\exists (n : \bN), \cod n \land f(n) = 0)}\]
\end{definition}

\begin{definition}[Explicit existential]
  An EFdata is said to have explicit existential if there is an evidenced $e$ such that for any system T type $\tau$ and any function $\varphi(-) : \tau \to \Prop$ there exists $a \in \tau$ such that
  \[\relEF{e}{\exists x^\tau, \cod x \land \varphi(x)}{\cod a \land \varphi(a)}\]
\end{definition}

Finally, we need to define a notion of morphism for an EFdata.

\begin{definition}[EFdata morphism]
  Let $\EFE_1,\EFE_2$ be two EFdata (each component of which will be written respectively $E_1,\Phi_1,\ldots$ and $E_2,\Phi_2,\ldots$). A morphism of EFdata is the data of two functions $F_{\Phi} : \Phi_1 \to \Phi2$ and $F_E : E_1 \to E_2$ (we will just write $F$) such that~:
  \begin{itemize}
  \item $\forall \varphi, \psi \in \Phi_1, \forall e \in E_1, \relEF{e}{\varphi}{\psi} \implies \relEF{F(e)}{F(\varphi)}{F(\psi)}$
  \item $F(\PropTop_1) = \PropTop_2$
  \item $F({\witTop}_1) = {\witTop}_2$
  \item $\forall \varphi, \psi \in \Phi_1, F({(\PropAnd \varphi \psi)}_1) = {(\PropAnd \varphi \psi)}_2$
  \item $\forall e, e' \in E_1, F({\witPair e {e'}}_1) = \witPair{F(e)}{F(e')}_2$
  \item $F({\witFst}_1) = {\witFst}_2$
  \item $F({\witSnd}_1) = {\witSnd}_2$
  \item $\forall \varphi \in \Phi_1, \forall \vec \psi \in \powerset(\Phi_1),F({(\PropImpl{\varphi}{\vec\psi})}_1) = {(\PropImpl{F(\varphi)}{\{F(\psi)\mid \psi \in \vec \psi\}})}_2$
  \item $\forall e \in E_1, F({\witLam e}_1) = {\witLam{F(e)}}_2$
  \item $F({\witEval}_1) = {\witEval}_2$
  \item $\exists f : \Phi_2 \to \Phi_1, \begin{cases}
      \exists \witx\in E_2, \forall \varphi \in \Phi_2,\relEF{\witx}{\varphi}{F(f(\varphi))}\\
      \exists \witx \in E_2, \forall \varphi \in \Phi_2, \relEF{\witx}{F(f(\varphi))}{\varphi}
  \end{cases}$
  \item $F({e_{0}}_1) = {e_{0}}_2$
  \item $F({e_{S}}_1) = {e_S}_2$
  \item $F({e_{\rec}}_1) = {e_{\rec}}_2$
  \item for all $n \in \bN$, $F_{\Phi}({\cod n}_1) = {\cod n}_2$
  \end{itemize}

  We define $\CatEFData$ as the category whose objects are EFdatas, and morphisms are those defined above.
\end{definition}

We can now state our generalized theorem about $\FT$.

\begin{theorem}\label{thm:main}
  Let $\EFE$ be an EFdata with unbounded search. Let $(W,\leq,\bOne)$ be an ordered set with lower bound $\bOne$ and $F : W \to \CatEFData$ be a functor such that $F(\bOne) = \EFE$. If the following two conditions are satisfied:
  \begin{itemize}
  \item there is a dense subset $W' \subseteq W$ such that for any $w \in W', F(w)$ has continuity and explicit existential.
  \item for any $w \in W$, function $B : (\List(\bN) \to \bB) \to \bB$ with an infinite path and a code in $w$, there is an element $w' \geq w$, a code $e_\alpha$ of an infinite path $\alpha$ in $B$ and a code $e \in E_{w'}$ such that for any $n \in \bN$, $\relEF{e}{\cod n}{\restr \alpha n \in B}$
  \end{itemize}
  then $\EFE \real \FT$.
\end{theorem}

\begin{proof}
  We apply the proof of theorem \ref{thm:FT} in the general setting. The proposition we want to prove is that there exists $\witx \in E$ such that
  \begin{multline*}
    \witx\real \forall B^{\List(\Bool) \to \Bool}, (\forall \alpha^{\{\Nat \to \Bool\}}, \exists n^{\{\Nat\}}, \restr \alpha n \in B) \implies\\
    \exists n^{\{\Nat\}}, \forall \alpha^{\{\Nat \to \Bool\}}, \restr \alpha n \in B
  \end{multline*}

  We consider $\List(\Bool)$ as an encoding by integer (every primitive function on this type is system T definable from $\Nat$). We assume there is $b$ witnessing the bar on $B$, and the definition of $C(b)$ from it is system T defined: this ensures that we can define the same function using the witnesses $\witZ,\witS,\witRec$. In the definition of $n_{C(b)}$, we can't define the fixpoint by itself, but we can define the function which, for any $n$, computes whether every $\ell \in \List(\Bool)$ of size $n$ is inside $C(b)$. Call this function $F_C : \Nat \to \Bool$. This function has a code $e_{F_C}$.

  By hypothesis, the evidenced frame has unbounded search, so we can apply it to $\lnot \circ F_C$ (because we search for a $n$ such that $F_C(n) = 1$). Combining it with the code $e_{F_C}$, this gives an evidenced of a uniform bound. Thus, all that is left to prove is that $\lnot \circ F_C$ indeed has a $0$.

  Suppose that $F_C(n) = 0$ for any $n$. This means that for any $n \in \bN$, there is a list of size $n$ avoiding $C$. Using Weak König's Lemma, we find a path $\alpha : \bN \to \bB$ such that $\restr \alpha n \notin C$ for any $n \in \bN$. By hypothesis on our functor $W \to \CatEFData$, we find $w' \in W$ and a code $e_\alpha \in \cod \alpha$ in $w'$. By the other hypothesis, we can assume (up to going to a later world) that $F(w')$ has continuity.

  Let $g$ be the morphism from $\EFE$ to $F(w')$. By construction of morphisms, we have that
  \[g(b)\real g(\forall \alpha^{\{\Nat \to \Bool\}}, \exists n^{\{\Nat\}}, \restr \alpha n \in B)\]
  but as $g$ commutes with $\cod -$, with $\implies$, and with quantification $\forall$, we deduce that
  \[g(b)\real \forall \alpha^{\{\Nat \to \Bool\}}, \exists n^{\{\Nat\}}, \restr \alpha n \in B\]
  hence, $g(b) e_{\alpha} \real \exists n^{\{\Nat\}}, \restr \alpha n \in g(B)$. Using the fact that $g(b)$ is a code of a function $(\bN \to \bN) \to \bN$, we can apply continuity to find $m$ such that $\restr \alpha m = \restr \beta m \implies g(b) \alpha = g(b) \beta$ for any $\beta : \bN \to \bN$ with a code. This means that we can apply the same argument as in theorem \ref{thm:FT} to have a contradiction~: taking $M$ the maximum of this $m$ and of the encoded $n^{\{\Nat\}}$ by $g(b) e_{\alpha}$ (that we can take by explicit existential), we get $\restr \alpha M \in C(b)$, which is contradicting the fact that $\alpha$ always avoids $C(b)$.
\end{proof}

\subsection{Application, limitations}

Now that we stated the theorem \ref{thm:main}, we give examples of a realizability model satisfying the conditions (and thus, realizing $\FT$).

The first example is the one given before. For each $\enviro \in \Oracle$, we can define the EFdata of $\SATE$ family as in proposition \ref{prop:EF_notre}. For $\enviro, \enviroT \in \Oracle$ such that $\enviro\infOr \enviroT$, we have a morphism of EFdata given by inclusion. For each $\enviro$, the $\SATE$ family EFdata has continuity, explicit existential and unbounded search, hence the fact that it satisfies $\FT$.

The second example is the historical one: the Kleene algebra $K_2$ satisfies the premisses. The preorder is the trivial one, with only one element. Continuity is by definition of the PCA (it is the set of continuous functions $(\bN \to \bN) \to \bN$), unbounded search can be defined for any PCA by encoding $\Theta$ (or the combinator $Y$) in combinatorial logic. Any function $\bN \to \bN$ has a code, hence the condition that a path exists for any decidable tree encoded. Finally, explicit existential is true for intuitionnistic PCAs. We thus find back the usual result that $K_2$ satisfies $\FT$.

A third example is the model we based our generalization on: the model by Lubarsky in [REF].

The existence of a path in a possible future is essentiel. Indeed, the $\lambda$-calculus version of the $K_1$ realizability algebra, given by the set of saturated set for the pure untyped $\lambda$-calculus, does not realize $\FT$. To show this, we use the Kleene tree. For any computable path, given by a code $e$, we can simulate $e$ for increasing time duration until the finite path gets out of the tree, which will always happen because the infinite path in the Kleene tree are all uncomputable. This thus gives a bar for the Kleene tree, but this bar can't be uniform as the length of lists in this tree is unbounded.

Finally, we advocate that the condition that we consider EFdatas and not just EF is not a limitation by itself. The additionnal data, indeed, always exist inside the realizability topos given by an EF.

Take an EF $(E,\Phi,\relEFdot)$. Using the UFam construction to get a tripos, and then using the tripos to topos construction, we can describe the objects of this topos. We focus on one of its full subcategories, given by the assemblies which are the following data:
\begin{itemize}
\item a set $X$
\item a function $E_X : X \to \Phi$
\end{itemize}
and morphisms $(X,E_X) \to (Y,E_Y)$ are given by
\begin{itemize}
\item a function $f : X \to Y$
\item an evidence $\relEF{e}{E_X(x)}{E_Y(f(x))}$ uniform on $x \in X$
\end{itemize}
This exactly corresponds to the way we define codes for system T types, so every construction in the EFdata can be translated inside the realizability topos. We are left to show that there is indeed an encoding $\bN \to \Phi$, with an evidence for $0$, the function $S$ and the recursor (which must be uniform for any codomain).

A FAIRE : Church encoding, justifier qu'on a un NNO dans le topos de réalisabilité

This means that any EF can be made into an EFdata by chosing some encoding of natural numbers.

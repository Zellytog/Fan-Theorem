\subsection{Weak Fan Theorem is realized}

Before proving that $\FT$ is realized, a continuity lemma is needed.

\begin{lemma}[Continuity]\label{lem:continuity}
  Let $\enviro\in\Oracle$ be a reduction context and $t \in \Lambda$ be a term such that
  \[t \realP{\enviro}{} (\Nat \to \Nat) \to \Nat\]
  Then for any $\alpha \realP{\enviro}{} \Nat \to \Nat$, there exists $n\in \bN$ such that
  \[\forall (\beta \realP{\enviro}{} \Nat \to \Nat), (\forall i < n, \beta\;\overline i \eqredE \alpha\;\overline i) \implies t\;\beta \eqredE t\;\alpha\]
\end{lemma}

The continuity is proved by analyzing carefully the reduction sequence from $t(\alpha)$: this reduction can call $\alpha$ only a finite number of times as the reduction itself is finite, so $\alpha$ returns a finite number of values, and any other function returning the same values on the same input will give the same reduction sequence.

Using this lemma and the $\lambda$-terms from the previous subsection, we can now realize $\FT$.

\begin{theorem}[Realization of Fan Theorem]\label{thm:FT}
  Using our previous definitions:
  \[\lambda b. n_{C_b}\;0 \realU \FT\]
\end{theorem}

\begin{proof}
  Let's prove that $\lambda b. n_{C_b}\;0 \realU\FT$. Let $\enviro\in\Oracle$ be a reduction context of length $m$ and
  \[b \realP{\enviro}{} \forall \alpha^{\{\Nat \to \Bool\}}, \exists n^{\{\Nat\}}, \restr \alpha n \in B\]
  for some function $B : \List(\bB) \to \SAT$ closed by extension. By saturation, it suffices to show that
  \[n_{C_b}\;0 \realP{\enviro}{} \exists n^{\{\Nat\}}, \forall \alpha^{\{\Nat \to \Bool\}}, \restr\alpha n \in B\]
  but we already know that $n_{C_b}\;0$ either reduces to an integer or diverges. If it reduces to an integer, then this integer is a uniform bound for $C$, but as $C \subseteq B$, it follows that $n_{C_b}\;0$ is a uniform bound for $B$.

  Now, suppose $n_{C_b}\;0$ diverges. This means that for any $n \in \bN$, there exists some word $w_n \in \bB^\star$ such that $w_n \notin C$. By applying Weak König's Lemma (in our meta-theory), we find an infinite path
  \[\alpha : \bN \to \bB\]
  such that for all $n \in \bN, \restr \alpha n \notin C$. We now enrich our reduction context with this alpha:
  \[\enviroT \defeq \enviro \smallfrown \alpha\]
  so that $\oracle_m \realP{\enviroT}{} (\Nat \to \Bool)(\alpha)$. By definition of $b$, we know that
  \[b\;\oracle_m\realP{\enviroT}{} \exists n^{\{\Nat\}}, \restr \alpha n \in B\]
  which we can destruct as some $n \in \bN$ such that
  \[\pi_1\;(b\;\oracle_m) \redRTP{\enviroT} n \qquad \pi_2\;(b\;\oracle_m)\realP{\enviroT}{} \restr \alpha n \in B\]
  Using lemma \ref{lem:continuity}, we can find an index $p\in \bN$ such that
  \[\forall (\beta \realP{\enviroT}{} \Nat \to \Bool), (\forall i < p, \beta\;\overline i \redRTP{\enviroT} \overline{\alpha(i)}) \implies \pi_1(b\;\beta) \redRTP{\enviroT} n\]

  Let
  \[M \defeq \max(n,p)\]
  by continuity, we know that $\pi_1\;(b\;\alpha_M) \redRTP{\enviroT}{} n$, so when computing $C_b\;\alpha_M$, there exists a list $\ell$ of length $\leq M$ such that $\restr\alpha{{\pi_1\;(b\;\ell 0^\infty)}}$ is a prefix of $\restr{\alpha}{M}$, so $\restr{\alpha}{M} \in C$. But by hypothesis, $\restr{\alpha}{M} \notin C$; hence a contradiction.

  This means that there is no case when $n_{C_b}$ diverges, so
  \[n_{C_b}\realP{\enviro}{} \exists n^{\{\Nat\}}, \forall \alpha^{\{\Nat \to \Bool\}}, \restr\alpha n \in B\]
  which means that $\lambda b.n_{C_b}\realU \FT$.
\end{proof}

Not only does the model satisfy $\FT$, but it also does not satisfy $\WKL$. By this, we mean that $\WKL\notin\ThRealU$, making our model a separating model between the two principles.

To prove that $\WKL$ is not realized, we use the Kleene tree construction (which can be relativized without issue). The tree, seen as a predicate over binary words, only contains those words which are different from any computable function in a finite time. For example, if $\ell$ is of length $n$, then we check whether $\ell_0$, its first value, is different from the value of $\Enum{}{0}(0)$ computed on $n$ steps (if it exists). This way, any finite path is computable, because a universal machine exists and can simulate the computation of a code on an input for a finite time. However, no infinite path can be found from a computable point of view: if $\alpha$ is an infinite path in $K$, this means that for any $\Enum{}{i}(i)\convcal$ (which means that it is computed in a finite time), we have that $\alpha_i \neq \Enum{}{i}(i)$, hence $\alpha$ is diagonaly non computable, hence non computable. In the realizability model, this implies that no path of $K$ is realized.

%\emnote{separation + eventuellement décrire ici l'arbre kleene}

\begin{theorem}
  There is no $t \in \Lambda$ such that $t \realU \WKL$.
\end{theorem}

\begin{proof}
  Suppose that there is $t \realU \WKL$. Let $\enviro\in\Oracle$, then $t \realP{}{\enviro} \WKL$. Let $\EnumE{}$ be an enumeration of $\sigma$-computable functions given by fixing a Gödel encoding of terms $t$. The Kleene tree $K_\enviro$ is defined as the set of lists $\ell \in \List(\Bool)$ such that for $i < |\ell|$, if $\EnumE{i}(i)\convcal[i]$ (meaning that the computation of $\EnumE{i}$ terminates in less than $i$ steps) then $\EnumE{i}(i) \neq \ell_i$.
  $K_\enviro$ is thus a tree with arbitrarily large finite sequences $\ell \in K$, but an finite path inside $K_\enviro$ cannot be $\enviro$-computable as it differs with every $\EnumE{i}$ on $i$. The realized functions $\Nat \to \Bool$ at context $\enviro$ being $\enviro$-computable, and taking a term $u$ encoding the premisses of $\WKL$ for $K_\enviro$ (the fact that $K_\enviro$ has arbitrarily large finite sequences), $t\;u$ should realize an infinite path in $K_\enviro$ at context $\enviro$, which is impossible.

  Hence $\WKL\notin\ThReal{\enviro}$, and $\WKL\notin\ThRealU$.
\end{proof}

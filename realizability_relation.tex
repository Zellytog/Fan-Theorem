\subsection{Realizability relation}

We now define the realizability relation $\real$. For this, we must first give a Heyting pre-algebra interpreting $\Prop$. This will be done, given a reduction context $\enviro$, by using $\SAT$. Then, we define the set of codes of any element $s \in \bS$, for a given sort $\bS$. This set will depend on $\enviro$, as oracles add new code.

As each $\enviro \in \Oracle$ gives a set of codes and a set of realizers for each formula, we consider as the Heyting pre-algebra the $\SAT$ family one.

\begin{definition}[Code of an element]
  Let $\sigma \in \Oracle$. We define by induction, for any $S \in \Sort$ and any element $s \in \bS$, the set $\codE s \in \SAT$:
  \begin{itemize}
  \item if $S = \Nat$, then for $n \in \bN$,
    \[\codE{n} \defeq \{ t \in \Lambda \mid t \redERT \overline n\}\]
  \item if $S = \Bool$, then
    \[\codE{\top} \defeq \{ t \in \Lambda \mid t \redERT \rtt\}\quad\text{ and }\quad\codE{\bot} \defeq \{ t \in \Lambda \mid t \redERT \rff\}\]
  \item if $S = \List(T)$, then
    \[\codE{\varepsilon} \defeq \{ t \in \Lambda \mid t \redERT \nil \}\]
    and for all $x \in \bT^\star$, $y \in \bT$,
    \[\codE{x \smallfrown y} \defeq \{ t \in \Lambda \mid \exists u \in \codE x, \exists v \in \codE y, t \redERT u \append v\}\]
  \item if $S = T \times U$, then for all $x \in \bT$, $y \in \bU$,
    \[\codE{(x,y)} \defeq \{ t \in \Lambda \mid (\pi_1\;t \in \codE x) \land (\pi_2\;t \in \codE y)\}\]
  \item if $S = T \to U$, then for all $f : \bT \to \bU$,
    \[\codE f \defeq \{ t \in \Lambda \mid \forall s \in \bT, \forall \enviroT \in \cyl{\enviro},\forall u \in \codP s \enviroT, tu \in \codP{f(s)}\enviroT\}\]
  \item if $S = \Prop$, then for all $(A_\enviro)_{\enviro \in \Oracle} \in \prod_{\enviro \in \Oracle}\SATE$,
    \[\codE{(A)_{\enviro}} \defeq A_\enviro\]
  \end{itemize}
\end{definition}

The fact that $\codE s \in\SATE$ is a straightforward induction on the set of sorts.

\begin{definition}[Realizability relation]
  Let $\enviro \in \Oracle$, $\Gamma \in \Hctx$, $\termenv \models \Gamma$ and $\Gamma\vdash \varphi : \Prop$. We define by induction the realizability relation:
  \begin{itemize}
  \item $t \realPP S(\termt) \defeq t \in \codE{\termenv(\termt)}$
  \item $t \realPP \varphi \to \psi \defeq \forall \enviroT \in \cyl{\enviro}, u \realP{\enviroT}{\termenv} \varphi \implies tu \realP{\enviroT}{\termenv}\psi$
  \item $t \realPP \varphi \land \psi \defeq (\pi_1\;t \realPP \varphi) \land (\pi_2\;t \realPP \psi)$
  \item $t \realPP \forall \varx^S, \varphi \defeq \forall s \in \bS, t\realP{\enviro}{\termenv[\varx \mapsto s]} \varphi$
  \item $t \realPP \exists \varx^S, \varphi \defeq \exists s \in \bS, t \realP{\enviro}{\termenv[\varx \mapsto s]} \varphi$
  \end{itemize}

  If $\termenv = \varnothing$ (in which case $\Gamma = \nil$), then we only write $t \realP{\enviro}{} \varphi$.

  We also define the universal realizability relation:
  \[t \realUP{\termenv} \varphi \defeq \forall \enviro \in \Oracle, t \realP{\enviro}{\termenv}\varphi\]
\end{definition}

In fact, $t \realUP{\termenv} \varphi$ can be defined just as $t \realP{}{\termenv} \varphi$, thanks to the following property.

\begin{proposition}\label{prop.monotonicity}
  let $\enviro, \enviroT \in \Oracle$, $\enviro \infOr \enviroT$, $\Gamma \in \Hctx, \termenv \models \Gamma, \Gamma \vdash \varphi : \Prop$. Then, for every $t$, we have the following implication:
  \[t \realPP \varphi \implies t \realP{\enviroT}{\termenv} \varphi\]
\end{proposition}

\begin{proof}
  We prove this by induction on $\varphi$ (and thus, for the case where $\varphi = S(\termt)$, by induction on $S$):
  \begin{itemize}
  \item if $S = \Nat$, then $t \realPP \Nat(\termt)$ means that $t \redERT \overline n$, thus $t \redRTP{\enviroT} \overline n$, so $t \realP{\enviroT}{\termenv} \Nat(\termt)$.
  \item if $S \in \{\Bool, \List(T), T \times U, \Prop\}$, the argument is the same.
  \item if $S = T \to U$, then suppose that $t \realPP \Nat(\termt)$. To prove the expected result, let $\enviroTT \in \cyl{\enviroT}$ and $u \in \codP{\termenv(\termt)} \enviroTT$, then $tu \in \codP{\termenv(\termt)}{\enviroTT}$ by hypothesis on $t$.
  \item if $\varphi = \psi \to \chi$, then let $\enviroTT \in \cyl{\enviroT}$ and $u \realP{\enviroTT}{\termenv} \psi$. By transitivity, $\enviro \infOr \enviroTT$, so by hypothesis on $t$, $tu \realP{\enviroTT}{\termenv} \psi$
  \item if $\varphi = \psi \land \chi$, then we can conclude directly by induction hypothesis
  \item the cases $\forall, \exists$ are the same: the induction hypothesis goes through without issue
  \end{itemize}
  Hence the monotonicity of realizability with regard to order on reduction contexts.
\end{proof}

\begin{notation}
  For a formula $\varphi$, we write
  \begin{align*}
    \realFE{\varphi} &\defeq \{ t \in \Lambda \mid t \realPP \varphi \}\\
    \realUFE{\varphi} &\defeq (\realFE{\varphi})_{\enviro \in \Oracle} \\
  \end{align*}
\end{notation}

\begin{proposition}
  For every $\enviro \in \Oracle$, $\Gamma \in \Hctx$, $\termenv \models \Gamma$ and $\Gamma \vdash \varphi : \Prop$, $\realFE{\varphi} \in \SAT$ and $\realUFE{\varphi} \in \HeytingFamily$.
\end{proposition}

\begin{proof}
  The proof is by induction on the realizability relation:
  \begin{itemize}
  \item if $\varphi = S(\termt)$, then we just use the fact that each $\codE{s}$ is in $\SAT$ (respectively that $\cod{s} \in \SAT$), so it's the case in particular for $\codE{\termenv(\termt)}$.
  \item if $\varphi = \psi \to \chi$, then let $t,u \in \Lambda, t \red u$ and $u \realPP \psi \to \chi$. Let's prove that $t \realPP \psi \to \chi$. Let $\enviroT \in \cyl{\enviro}$ and $v \realP{\enviroT}{\termenv} \psi$, let's prove that $tv \realP{\enviroT}{\termenv} \chi$. By compatibility, $tv \red uv$, but as $u \realPP \psi \to \chi$, this means that $uv \realP{\enviroT}{\termenv} \chi$, hence (as $\realFP{\chi}{\termenv}{\enviroT}$ is saturated by induction hypothesis) $tv \realP{\enviroT}{\termenv} \chi$.
  \item if $\varphi = \psi \land \chi$, then let $t,u \in \Lambda, t \red u$ and $u \realP \psi \land \chi$. This means that both $\pi_1\;u \realPP \psi$ and $\pi_2\;u \realPP \chi$, but then $\pi_i\;t \red \pi_i\;u$ for $i \in \{1,2\}$, so by saturation of $\realFE{\psi}$ and $\realFE{\chi}$ (given by induction hypotheses), $\pi_1\;t \realPP \psi$ and $\pi_2\;t \realPP \chi$. Thus $t \realPP \psi \land \chi$.
  \item if $\varphi = \forall \varx^S, \varphi$, then we use the fact that $\SAT$ is closed by intersection.
  \item if $\varphi = \exists \varx^S, \varphi$, then we use the fact that $\SAT$ is closed by union.
  \end{itemize}
  Hence the result (the monotonicity for $\realUFE{\varphi}$ is already proved in proposition \ref{prop.monotonicity}).
\end{proof}

\begin{notation}
  We define the relativised universal quantification of a proposition, $\forall \varx^{\{S\}}, \varphi$, as
  \[\forall \varx^{\{S\}}, \varphi \defeq \forall \varx^S, S(\varx) \to \varphi\]
  Thus, when realizing a formula like $\forall \varx^{\{S\}}, \varphi$, it suffices to suppose that there is an objet $s \in \bS$ with a code $t$ and t prove $\varphi$ using the data $t$.

  The relativised existantial quantification is defined by
  \[\exists \varx^{\{S\}}, \varphi \defeq \exists \varx^{S}, S(\varx) \land \varphi\]
  This type is naturally equiped with both
  \[\pi_1 \realU \varphi \land \psi \to \varphi \qquad \pi_2 \realU \varphi \land \psi \to \psi\]
  This means that, if $t\realP{\enviro}{\termenv} \exists\varx^{\{S\}}, \varphi(\varx)$, then there exists some $s \in \bS$ such that $\pi_1\;t$ is a code of $s$ (for the reduction contexte $\enviro$) and $\pi_2\;t \realP{\enviro}{\termenv[\vary\mapsto s]} \varphi(\vary)$ with $\vary$ a fresh variable.
\end{notation}

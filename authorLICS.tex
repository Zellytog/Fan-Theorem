\author{Titouan Leclercq}
\email{titouan.leclercq@univ-amu.fr}
% \orcid{1234-5678-901}
\affiliation{%
  \institution{Aix Marseille University, CNRS, I2M}
  \city{Marseille}
  \country{France}
}

\author{\'{E}tienne Miquey}
\email{etienne.miquey@univ-amu.fr	}
\orcid{0000-0002-5987-6547}
\affiliation{%
  \institution{Aix Marseille University, CNRS, I2M}
  \city{Marseille}
  \country{France}
}

%%
%% The abstract is a short summary of the work to be presented in the
%% article.
\begin{abstract}
  Friedman-Simpson’s original program of reverse mathematics,
as is also the case for most of standard mathematics,
has been developed in classical subsystems of second-order
arithmetic. As such, (classical) reverse mathematics
presents various limitations from a constructive point of view,
since for instance they are unable do distinguish between
a statement and its contrapositive (e.g. dependent choice and
the bar induction principles).
The case of Weak König Lemma (WKL) and Fan Theorem (FT)
is particularly interesting in that regard:
while WKL is well-known to implies FT, and if constructivists like
Brouwer rejected the former while admitting the latter,
the converse implication has not been much studied for years.
It is only recently that a growing enthusiasm for constructive reverse
mathematics pushed towards a finer-grained analysis of the connection
between such principles.
In addition to intuitionnistic reverse mathematics, the realizability approach
to logical principles adds a computational meaning to purely logical statements.
We follow this approach by giving a computational meaning to Brourwer's Fan
Theorem: building on recent work by Lubarsky and Rathjen, we construct a
realizability model of higher order logic based on $\lambda$-calculus, from
which we build up an abstract description of computational conditions entailing
FT.

%Building on recent work by Lubarsky and Rathjen,
%we give here a computational perspective on the separation between
%FT and WKL, by first providing a realizability interpretation of second-order
%logic based on the $\lambda$-calculus that validates Brouwer's Fan Theorem
%while negating Weak König Lemma.

\end{abstract}

%%
%% The code below is generated by the tool at http://dl.acm.org/ccs.cfm.
%% Please copy and paste the code instead of the example below.
% %%
% \begin{CCSXML}
% <ccs2012>
%  <concept>
%   <concept_id>00000000.0000000.0000000</concept_id>
%   <concept_desc>Do Not Use This Code, Generate the Correct Terms for Your Paper</concept_desc>
%   <concept_significance>500</concept_significance>
%  </concept>
%  <concept>
%   <concept_id>00000000.00000000.00000000</concept_id>
%   <concept_desc>Do Not Use This Code, Generate the Correct Terms for Your Paper</concept_desc>
%   <concept_significance>300</concept_significance>
%  </concept>
%  <concept>
%   <concept_id>00000000.00000000.00000000</concept_id>
%   <concept_desc>Do Not Use This Code, Generate the Correct Terms for Your Paper</concept_desc>
%   <concept_significance>100</concept_significance>
%  </concept>
%  <concept>
%   <concept_id>00000000.00000000.00000000</concept_id>
%   <concept_desc>Do Not Use This Code, Generate the Correct Terms for Your Paper</concept_desc>
%   <concept_significance>100</concept_significance>
%  </concept>
% </ccs2012>
% \end{CCSXML}

% \ccsdesc[500]{Do Not Use This Code~Generate the Correct Terms for Your Paper}
% \ccsdesc[300]{Do Not Use This Code~Generate the Correct Terms for Your Paper}
% \ccsdesc{Do Not Use This Code~Generate the Correct Terms for Your Paper}
% \ccsdesc[100]{Do Not Use This Code~Generate the Correct Terms for Your Paper}

%%
%% Keywords. The author(s) should pick words that accurately describe
%% the work being presented. Separate the keywords with commas.
% \keywords{Fan Theorem, Weak \Konig's Lemma, realizability, $\lambda$-calculus, oracles, reverse mathematics, continuity}

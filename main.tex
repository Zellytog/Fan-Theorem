% \providecommand{\mode}{-1} %% Neutre
% \providecommand{\mode}{0} %% FSCD
\providecommand{\mode}{1} %% LICS



\newcommand\switch[3]{%
  \ifcase\mode\relax
    #1
  \or
    #2
  \or
\else
    #3
  \fi
}
\newcommand\iflics[2]{\ifcase\mode\relax #2 \or #1 \else #2\fi}


\switch{
\documentclass[a4paper,USenglish,cleveref, autoref, thm-restate,authorcolumns]{lipics-v2021}
\newcommand{\vlong}[1]{#1}
\newcommand{\vshort}[1]{}
\pdfoutput=1 %uncomment to ensure pdflatex processing (mandatatory e.g. to submit to arXiv
\hideLIPIcs  %uncomment to remove references to LIPIcs series (logo, DOI, ...), e.g. when preparing a pre-final version to be uploaded to arXiv or another public repository
%\graphicspath{{./graphics/}}%helpful if your graphic files are in another directory
% \usepackage[T1]{fontenc}
\usepackage{prelude-FSCD}


\usepackage{dsfont}
\usepackage[draft,inline,nomargin]{fixme}
\fxusetheme{color}
\fxuseenvlayout{color}
\FXRegisterAuthor{tl}{tlg}{\color{teal}[T]}
\FXRegisterAuthor{em}{aem}{\color{blue}[É]}

% \usepackage[colorlinks=true]{hyperref}
%%%% REFS
\Crefname{equation}{Eq.}{Eqs.}
\Crefname{figure}{Fig.}{Figs.}
\Crefname{tabular}{Tab.}{Tabs.}
\Crefname{section}{Sec.}{Secs.}
\Crefname{definition}{Def.}{Defs.}
\Crefname{defi}{Def.}{Defs.}
\Crefname{lemma}{Lem.}{Lems.}
\Crefname{lem}{Lem.}{Lems.}
\Crefname{theorem}{Thm.}{Thms.}
\Crefname{thm}{Thm.}{Thms.}
\Crefname{paragraph}{Sec.}{Secs.}
\Crefname{appendix}{Appx.}{Appxs.}
\Crefname{corollary}{Cor.}{Cors.}
\Crefname{example}{Ex.}{Exs.}
\Crefname{proposition}{Prop.}{Props.}
\Crefname{remark}{Rem.}{Rems.}
% \usepackage{ebproof}

\title{}
\authorrunning{Leclercq, Miquey}
\Copyright{Leclercq and Miquey} %TODO mandatory, please use full first names. LIPIcs license is "CC-BY";  http://creativecommons.org/licenses/by/3.0/
\author{Titouan Leclercq}{Aix Marseille University, CNRS, I2M, Marseille, France}{titouan.leclercq@univ-amu.fr	}{}{}
\author{\'{E}tienne Miquey}{Aix Marseille University, CNRS, I2M, Marseille, France}{etienne.miquey@univ-amu.fr	}{0000-0002-5987-6547}{}
\ccsdesc[500]{Theory of computation}
%\ccsdesc[500]{Theory of computation~Proof theory}
% \ccsdesc[300]{Theory of computation~Linear logic}
\keywords{Combinatory algebras, Monads, Effects, Realizability, Evidenced frames} %TODO mandatory; please add comma-separated list of keywords
\funding{ANR blabla}
\acknowledgements{}
\nolinenumbers %uncomment to disable line numbering




% 
%%%%%%%%%%%%%%%%%%%%%%%%%%%%%
%----- DEFINITION LIKE -----%
%%%%%%%%%%%%%%%%%%%%%%%%%%%%%

\newtheoremstyle{mydefinitionstyle} % Name
{}          % Space above
{}          % Space below
{}          % Body font
{}          % Indent amount
{\bfseries} % Theorem head font
{.}         % Punctuation after theorem head
{ }         % Space after theorem head, ' ', or \newline
{\thmname{#1}\thmnumber{ #2}\thmnote{ (#3)}} % Theorem head spec (can be left empty, meaning `normal')

\theoremstyle{mydefinitionstyle}

\newtheorem{definition}{Definition}[subsection]
\newtheorem{notation}[definition]{Notation}
\newtheorem{remark}[definition]{Remark}
\newtheorem{exercise}[definition]{Exercise}
\newtheorem{problem}[definition]{Problem}


%%%%%%%%%%%%%%%%%%%%%%%%%%
%----- THEOREM LIKE -----%
%%%%%%%%%%%%%%%%%%%%%%%%%%

\newtheoremstyle{mytheoremstyle} % Name
{}          % Space above
{}          % Space below
{\slshape}  % Body font
{}          % Indent amount
{\bfseries} % Theorem head font
{.}         % Punctuation after theorem head
{ }         % Space after theorem head, ' ', or \newline
{\thmname{#1}\thmnumber{ #2}\thmnote{ (#3)}} % Theorem head spec (can be left empty, meaning `normal')

\theoremstyle{mytheoremstyle}

\newtheorem{property}[definition]{Property}
\newtheorem{proposition}[definition]{Proposition}
\newtheorem{claim}[definition]{Claim}
\newtheorem{lemma}[definition]{Lemma}
\newtheorem{theorem}[definition]{Theorem}
\newtheorem{corollary}[definition]{Corollary}
\newtheorem{axiom}[definition]{Axiom}
\newtheorem{conjecture}[definition]{Conjecture}



%%%%%%%%%%%%%%%%%%%%%%%%%%
%----- EXAMPLE LIKE -----%
%%%%%%%%%%%%%%%%%%%%%%%%%%

\makeatletter
\newcommand{\newexample}[2]{%
	\newenvironment{#1}[1][]{%
		\par%
		\def \param{##1}%
		\normalfont \topsep6\p@\@plus6\p@\relax%
		\trivlist%
		\item \relax%
		{%
			\itshape%
			#2%
			\ifx\param\empty%
				\relax%
			\else%
				\ (\param)%
			\fi%
			\@addpunct{.}
		}%
		\hspace%
		\labelsep%
		\ignorespaces%
	}{%
		\endtrivlist\@endpefalse
	}
}
\makeatother

\newexample{example}{Example}
\newexample{solution}{Solution}
\newexample{question}{Question}
\newexample{answer}{Answer}

\newtheorem{notation}[definition]{Notation}

}{
\input{header_lics}
}{
    \documentclass[11pt, a4paper, english]{article}
    \usepackage{prelude}
    
%%%%%%%%%%%%%%%%%%%%%%%%%%%%%
%----- DEFINITION LIKE -----%
%%%%%%%%%%%%%%%%%%%%%%%%%%%%%

\newtheoremstyle{mydefinitionstyle} % Name
{}          % Space above
{}          % Space below
{}          % Body font
{}          % Indent amount
{\bfseries} % Theorem head font
{.}         % Punctuation after theorem head
{ }         % Space after theorem head, ' ', or \newline
{\thmname{#1}\thmnumber{ #2}\thmnote{ (#3)}} % Theorem head spec (can be left empty, meaning `normal')

\theoremstyle{mydefinitionstyle}

\newtheorem{definition}{Definition}[subsection]
\newtheorem{notation}[definition]{Notation}
\newtheorem{remark}[definition]{Remark}
\newtheorem{exercise}[definition]{Exercise}
\newtheorem{problem}[definition]{Problem}


%%%%%%%%%%%%%%%%%%%%%%%%%%
%----- THEOREM LIKE -----%
%%%%%%%%%%%%%%%%%%%%%%%%%%

\newtheoremstyle{mytheoremstyle} % Name
{}          % Space above
{}          % Space below
{\slshape}  % Body font
{}          % Indent amount
{\bfseries} % Theorem head font
{.}         % Punctuation after theorem head
{ }         % Space after theorem head, ' ', or \newline
{\thmname{#1}\thmnumber{ #2}\thmnote{ (#3)}} % Theorem head spec (can be left empty, meaning `normal')

\theoremstyle{mytheoremstyle}

\newtheorem{property}[definition]{Property}
\newtheorem{proposition}[definition]{Proposition}
\newtheorem{claim}[definition]{Claim}
\newtheorem{lemma}[definition]{Lemma}
\newtheorem{theorem}[definition]{Theorem}
\newtheorem{corollary}[definition]{Corollary}
\newtheorem{axiom}[definition]{Axiom}
\newtheorem{conjecture}[definition]{Conjecture}



%%%%%%%%%%%%%%%%%%%%%%%%%%
%----- EXAMPLE LIKE -----%
%%%%%%%%%%%%%%%%%%%%%%%%%%

\makeatletter
\newcommand{\newexample}[2]{%
	\newenvironment{#1}[1][]{%
		\par%
		\def \param{##1}%
		\normalfont \topsep6\p@\@plus6\p@\relax%
		\trivlist%
		\item \relax%
		{%
			\itshape%
			#2%
			\ifx\param\empty%
				\relax%
			\else%
				\ (\param)%
			\fi%
			\@addpunct{.}
		}%
		\hspace%
		\labelsep%
		\ignorespaces%
	}{%
		\endtrivlist\@endpefalse
	}
}
\makeatother

\newexample{example}{Example}
\newexample{solution}{Solution}
\newexample{question}{Question}
\newexample{answer}{Answer}

    \author{}
}



\begin{document}
\title{Something}
\iflics{\author{Titouan Leclercq}
\email{titouan.leclercq@univ-amu.fr}
% \orcid{1234-5678-901}
\affiliation{%
  \institution{Aix Marseille University, CNRS, I2M}
  \city{Marseille}
  \country{France}
}

\author{\'{E}tienne Miquey}
\email{etienne.miquey@univ-amu.fr	}
\orcid{0000-0002-5987-6547}
\affiliation{%
  \institution{Aix Marseille University, CNRS, I2M}
  \city{Marseille}
  \country{France}
}

%%
%% The abstract is a short summary of the work to be presented in the
%% article.
\begin{abstract}
  Friedman-Simpson’s original program of reverse mathematics,
as is also the case for most of standard mathematics,
has been developed in classical subsystems of second-order
arithmetic. As such, (classical) reverse mathematics
presents various limitations from a constructive point of view,
since for instance they are unable do distinguish between
a statement and its contrapositive (e.g. dependent choice and
the bar induction principles).
The case of Weak König Lemma (WKL) and Fan Theorem (FT)
is particularly interesting in that regard:
while WKL is well-known to implies FT, and if constructivists like
Brouwer rejected the former while admitting the latter,
the converse implication has not been much studied for years.
It is only recently that a growing enthusiasm for constructive reverse
mathematics pushed towards a finer-grained analysis of the connection
between such principles.
In addition to intuitionnistic reverse mathematics, the realizability approach
to logical principles adds a computational meaning to purely logical statements.
We follow this approach by giving a computational meaning to Brourwer's Fan
Theorem: building on recent work by Lubarsky and Rathjen, we construct a
realizability model of higher order logic based on $\lambda$-calculus, from
which we build up an abstract description of computational conditions entailing
FT.

%Building on recent work by Lubarsky and Rathjen,
%we give here a computational perspective on the separation between
%FT and WKL, by first providing a realizability interpretation of second-order
%logic based on the $\lambda$-calculus that validates Brouwer's Fan Theorem
%while negating Weak König Lemma.

\end{abstract}

%%
%% The code below is generated by the tool at http://dl.acm.org/ccs.cfm.
%% Please copy and paste the code instead of the example below.
% %%
% \begin{CCSXML}
% <ccs2012>
%  <concept>
%   <concept_id>00000000.0000000.0000000</concept_id>
%   <concept_desc>Do Not Use This Code, Generate the Correct Terms for Your Paper</concept_desc>
%   <concept_significance>500</concept_significance>
%  </concept>
%  <concept>
%   <concept_id>00000000.00000000.00000000</concept_id>
%   <concept_desc>Do Not Use This Code, Generate the Correct Terms for Your Paper</concept_desc>
%   <concept_significance>300</concept_significance>
%  </concept>
%  <concept>
%   <concept_id>00000000.00000000.00000000</concept_id>
%   <concept_desc>Do Not Use This Code, Generate the Correct Terms for Your Paper</concept_desc>
%   <concept_significance>100</concept_significance>
%  </concept>
%  <concept>
%   <concept_id>00000000.00000000.00000000</concept_id>
%   <concept_desc>Do Not Use This Code, Generate the Correct Terms for Your Paper</concept_desc>
%   <concept_significance>100</concept_significance>
%  </concept>
% </ccs2012>
% \end{CCSXML}

% \ccsdesc[500]{Do Not Use This Code~Generate the Correct Terms for Your Paper}
% \ccsdesc[300]{Do Not Use This Code~Generate the Correct Terms for Your Paper}
% \ccsdesc{Do Not Use This Code~Generate the Correct Terms for Your Paper}
% \ccsdesc[100]{Do Not Use This Code~Generate the Correct Terms for Your Paper}

%%
%% Keywords. The author(s) should pick words that accurately describe
%% the work being presented. Separate the keywords with commas.
% \keywords{Fan Theorem, Weak \Konig's Lemma, realizability, $\lambda$-calculus, oracles, reverse mathematics, continuity}
}{}

\maketitle
\iflics{}{\begin{abstract}
 Friedman-Simpson’s original program of reverse mathematics,
as is also the case for most of standard mathematics,
has been developed in classical subsystems of second-order
arithmetic. As such, (classical) reverse mathematics
presents various limitations from a constructive point of view,
since for instance they are unable do distinguish between
a statement and its contrapositive (e.g. dependent choice and
the bar induction principles).
The case of Weak König Lemma (WKL) and Fan Theorem (FT)
is particularly interesting in that regard:
while WKL is well-known to implies FT, and if constructivists like
Brouwer rejected the former while admitting the latter,
the converse implication has not been much studied for years.
It is only recently that a growing enthusiasm for constructive reverse
mathematics pushed towards a finer-grained analysis of the connection
between such principles.
In addition to intuitionnistic reverse mathematics, the realizability approach
to logical principles adds a computational meaning to purely logical statements.
We follow this approach by giving a computational meaning to Brourwer's Fan
Theorem: building on recent work by Lubarsky and Rathjen, we construct a
realizability model of higher order logic based on $\lambda$-calculus, from
which we build up an abstract description of computational conditions entailing
FT.

%Building on recent work by Lubarsky and Rathjen,
%we give here a computational perspective on the separation between
%FT and WKL, by first providing a realizability interpretation of second-order
%logic based on the $\lambda$-calculus that validates Brouwer's Fan Theorem
%while negating Weak König Lemma.

\end{abstract}}


\section{Introduction}




% \paragraph{Reverse maths}
The program of reverse mathematics, first initiated by Friedman in the 70s,
aims at answering questions of the form: ``\emph{what are the minimal axioms necessary to derive the theorem $T$?}''.
To do so, formulas are classified via their logical equivalence classes over the ambient theory $\T$, which has then to be chosen carefully.
Indeed, $\T$ has to be expressive enough for usual mathematical statements to have meaning, while a too strong theory would identify all the provable formulas. Classical reverse mathematics has now identified a well-
established hierarchy of subsystems of second-order arithmetic serving this purpose~\cite{Simpson09}, but the literature in constructive reverse mathematics is much more recent and the situation is more complex: several principles compatible with constructive mathematics lead to logical inconsistencies when added together (\emph{e.g.}, Church’s thesis and the law of excluded-middle).





\paragraph{Constructive reverse maths}
% \emnote{Choice principles,Brede-Herbelin, intuitionistic logic}
Among the various limitations that classical reverse mathematics present
from a constructive point of view,  they are for instance intrinsically
bound to classify theorems that are compatible with classical logic (which is not the case for all intuitionistic theorems), or unable do distinguish between a statement and its contrapositive.
More recently, Ishihara advocated for a constructive approach to reverse
mathematics to overcome such limitations~\cite{Ishihara06}, but with
purely logical considerations, without paying attention to the computational counterpart of constructive results.
It is only during the last few years that works in that direction were developed, especially around Gödel's completeness theorem for first-order logic~\cite{HerIli22,ForKirWeh21}.
We advocate for a generalization of this approach, which has shown to be very conducive over the past few years~\cite{KirZen25,CohEtAl24,HerKir24,Bauer22}.
% is tailored for that purpose, by tackling first the problem of capturing the
% precise computational counterpart of choice principles.


\paragraph{Choice principles}
In a recent paper, Brede \& Herbelin put forward a generalized variant of the dependent choice $\GDC_{ABT}$, and its contrapositive bar induction principle $\GBI_{ABT}$, expressed in terms of tree~\cite{BreHer21}.
% named $\GDC_{ABT}$ and $\GBI_{ABT}$.
Different instantiations of the parameters $A$, $B$, $T$, which refer to the objects they apply to, allow to capture
a structured hierarchy of choice principles, ranging from Weak K\"onig's Lemma
(when $A$ is the type $\bN$ of natural numbers and $B$ is the type $\mathbb{B}$ of Booleans)
to the relational variant of the Axiom of Choice
(when $T$ is a filter induced by the considered relation), together with their contrapositive.
Following the recent advances in constructive reverse mathematics, we suggest to consider these principles with a computational point of view, and especially the question: ``\emph{what is the computational content of each of these principles?}''






\paragraph{Realizability interpretations}
Such a question raises several difficulties.

a/ how to relate more complex with logical principles with computational features?
following Krivine: new reasoning principles = extra computational features

b/ how to prove that we capture excatly the computational content:
1: enough? for instance full AC trivial in purely intuitionistic models àla BHK.  This is due do an implicit restriction, that computations in $A\to B$ are functions, which does not survive to extension. Several work fors AC$_\bN$/ DC
to recover it via memoization techniques~\cite{BerBezCoq98,Herbelin12,Miquey18a,CohFarTat19}.

2: not too much? of course, in such settings, AC$_\bN$ entails KL etc... we need to establish separation results: models A but not B

3: robustness: how to make sure this is not peculiar to the speficics of the implementation of the model ? keep memoization, change $\lambda$-calculus for a different computational system

-> EF: algebraic approach, compatible with effects~\cite{CohMiqTat21,CohGruKirMiq25mca}, also to abstract and give ``robust'' interpretation, statement as: ``any computational system featuring this and that induces a realizability interpretation satisfying ...''. So far, done for countable and dependent choice, related to memoization.







\paragraph{Fan Theorem \& Weak K\"onig's Lemma}


In this paper, we will focus and the lowest principles of Brede-Herbelin hierarchy, namely the Fan Theorem.
Constructive reverse maths are much more subtle than classical ones,
and the situation is already quite amusing: FT equivalent FT binary, while the contrapositive of the first, KL, is strictly stronger in a classical setting than WKL, the contrapositive of the second~\cite{FujiwaraKonigsLemmaWeak2021}.
This explains that even though a principle such as Brouwer's continuity principle was first stated more than a century ago~\cite{Brouwer1919}, continuity principles and their consequences, in particular FT, are still actively studied in constructive settings~\cite{VanAttenVanDalen2002,VeldmanBrouwer2022,BergerDecomposition2009,FujiwaraExtensionEquivalenceBrouwers2022,FujiwaraChoicePrinciplesCharacterizing2025,BaiEtAl25,CohRah23}.




In this work, we aim at :
providing a computational interpretation of FT,
+ identifying sufficient computational features via a robust approach,  while separating FT from WKL.


inspired from Lubarsky \& Rathjen's work, describing a computational ...
%
\cite{LubRat13}




\emnote{parler d'oracles}




\paragraph{Contributions}
orga = Contributions




\newcommand{\ZFC}{ZFC}
\newcommand{\ZF}{ZF}
\newcommand{\KL}{KL}
\newcommand{\RCAO}{RCA0}
\newcommand{\RCA}{RCA}
\newcommand{\double}{double}
\newcommand{\intt}{int}
% \newcommand{\FT}{FT}
%
%
% Les math\'ematiques usuelles se d\'eroulent dans $\ZFC$. Dans cette pratique des math\'ematiques, l'objectif est de d\'emontrer un maximum de r\'esultats, sans consid\'eration pour les moyens logiques employ\'es (sous r\'eserve que ceux-ci soient coh\'erents). Le cadre de ce rapport, celui des math\'ematiques à rebours, cherche au contraire à \'etudier la force logique de r\'esultats d\'ejà d\'emontr\'es~: quels principes doit-on admettre pour les d\'emontrer ?
%
% Cette question est renouvel\'ee dans le cadre de la logique intuitionniste et des math\'ematiques constructivistes (qui n'acceptent que les proc\'ed\'es d\'emonstratifs explicites), puisque l'on peut \'etudier le caractère constructif des principes en plus de leur force logique. Dans cette \'etude, une famille de formules a un rôle privil\'egi\'e : les principes de choix. Leur forme g\'en\'erale est
% \[\forall x^A, \exists y^B, R(x,y) \implies \exists f^{A\to B}, \forall x^A, R(x,f(x))\]
% Où $R$ est une relation binaire entre $A$ et $B$. Suivant le choix de $A,B$ et des formes que peut prendre $R$, on obtient de nombreux principes de choix. On peut, de plus, \'etudier une variante de ces principes en en prenant la contrapos\'ee~: en logique intuitionniste, ces ``principes de co-choix'' sont plus faibles, souvent strictement.
%
% Dans ce rapport, nous \'etudierons l'un des principes les plus faibles~: le lemme de Kőnig et sa contrapos\'ee, le Fan theorem. Le premier \'enonce qu'un arbre infini qui reste finiment branchant (comprendre par-là qu'à un n\oe ud donn\'e il n'y a qu'un nombre fini de fils directs) possède une branche infini ; le deuxième \'enonce que pour un arbre $T$, si toute branche infinie $\alpha$ sort de l'arbre à partir d'une longueur $n_\alpha$, alors on peut trouver un $n$ uniforme tel que toute branche infini $\alpha$ sort de $T$ à partir de la longueur $n$~: autrement dit, l'arbre a une hauteur finie. Sans rentrer dans les d\'etails, en consid\'erant un arbre comme une partie de $\bN^*$ close par pr\'efixe (son compl\'ementaire est donc une partie de $\bN^*$ close par extension), on peut \'enoncer le lemme de Kőnig et le Fan theorem de la façon suivante (en omettant plusieurs hypothèses sur $T$, respectivement $C$)~:
% \[\KL \defeq (\forall n^{\bN}, \exists p^{\bN^*}, |p| = n \land p \in T) \implies \exists \alpha^{\bN\to\bN}, \forall n^{\bN}, \alpha_0\ldots\alpha_{n-1}\in T\]
% \[\FT \defeq (\forall \alpha^{\bN\to\bN}, \exists n^\bN, \alpha_0\ldots \alpha_{n-1}\in C)\implies \exists n^\bN, \forall \alpha^{\bN\to\bN}, \alpha_0\ldots\alpha_{n-1}\in T\]
% L'\'ecriture $\exists p^{\bN^*},|p| = n \land p \in T$ est \'equivalente à $\exists \alpha^{\bN\to\bN}, \alpha_0\ldots\alpha_{n-1}\in T$, mais plus naturelle à consid\'erer. On suppose ici que $T$ est clos par pr\'efixe et $C$ par extension, nous donnant donc que $\FT$ est la contrapos\'ee de $\KL$.
%
% Mentionnons que $\KL$ comme $\FT$ sont toujours vrais dans $\ZFC$, et même dans $\ZF$~: on travaille donc dans une th\'eorie plus faible dans laquelle on peut \'etudier l'ajout de $\FT$ ou $\KL$ comme non trivial. La th\'eorie habituellement utilis\'ee en math\'ematiques à rebours est un fragment faible de l'arithm\'etique du second ordre nomm\'ee $\RCAO$. Cette th\'eorie est très expressive car elle permet de d\'ecrire les r\'eels ainsi que les fonctions continues voire mesurables. Cependant, cette expressivit\'e se fait au coût d'une grande quantit\'e de codages, qui nous int\'eressent peu. Nous faisons donc le choix de travailler dans une version de l'arithm\'etique dans laquelle on peut parler de façon interne de fonctions, de paires et d'autres outils primitifs en plus des entiers, le tout avec une logique du second ordre.
%
% Maintenant que le contexte est plus clair, nous pouvons d\'ecrire l'objectif du stage pr\'esent\'e dans ce rapport~: il s'agit de mesurer le contenu logique et calculatoire de $\FT$, en le s\'eparant notamment de $\KL$. Pour ce faire, nous utilisons la r\'ealisabilit\'e, qui est un outil reliant les langages de programmation et la logique.
%
% Depuis la d\'ecouverte de la correspondance de Curry-Howard, en effet, le lien entre les langages de programmation et la logique ont \'et\'e longuement explor\'es. Cette correspondance \'etablit un parallèle entre, d'un côt\'e, les programmes d'un langage typ\'e (par exemple un programme doublant sont entr\'ee, qu'on pourrait \'ecrire $\double : \intt \to \intt$) et les preuves d'une proposition. A une proposition $A$ il est possible de faire correspondre un type $A'$, et les programmes de types $A'$ correspondent alors à des preuves de $A$. La r\'ealisabilit\'e \'etend cette correspondance en consid\'erant, plutôt que la relation de typage, syntaxique, une approche s\'emantique. On entend par-là que la relation de typage est d\'efinie par un ensemble simple de règles d'inf\'erences, que l'on peut v\'erifier m\'ethodiquement, et que v\'erifier si un certain programme est d'un type donn\'e est en g\'en\'eral un problème d\'ecidable. La r\'ealisabilit\'e, elle, remplace la relation de typage $t : A'$ par une relation, directement entre $t$ et $A$ (la proposition), que l'on note $t \real A$, et qui signifie que $t$ est une preuve de $A$. Dans cette d\'efinition, on autorise les termes à être par exemple non typ\'es, et le seul critère consid\'er\'e pour dire que $t \real A$ est le comportement (la s\'emantique) de $t$. Avec cette relation, il devient alors possible de d\'efinir une nouvelle notion de modèle~: plutôt que d'attribuer à une proposition une valeur de v\'erit\'e dans $\{0,1\}$, la valeur de v\'erit\'e de $A$ devient l'ensemble de ses preuves, qu'on appellera ses r\'ealiseurs.
%
% Partant d'un langage simple (le $\lambda$-calcul), nous allons construire un modèle de r\'ealisabilit\'e v\'erifiant Fan theorem mais invalidant le lemme de Kőnig (sous une forme faible). La preuve que ce modèle v\'erifie $\FT$ pourra ensuite servir à chercher une pr\'esentation plus g\'en\'erale de l'argument qu'une forme de continuit\'e des fonctions calculables implique $\FT$, qui semble être une notion de continuit\'e des fonctions. En effet, il existe des cadres de travail plus g\'en\'eraux pour parler de r\'ealisabilit\'e, et extraire l'argument derrière la v\'erification de $\FT$ nous permettrait alors, en employant un de ces cadres de travail (en particulier les evidenced frames d\'ecrites dans \cite{CohMiqTat21}), d'extraire la notion calculatoire qui permet de v\'erifier ou non $\FT$.
%
% La principale contribution de ce stage, qui explique d'ailleurs le point de vue d\'evelopp\'e dans ce rapport, est la d\'efinition d'un modèle de r\'ealisabilit\'e pour aborder la r\'ealisabilit\'e permettant, plus tard, d'\'etudier les principes de choix. La construction de modèles de r\'ealisabilit\'e est un exercice classique~: on peut par exemple retrouver plusieurs modèles dans \cite{Dinis_2023} ou dans \cite{COHEN201987}, mais ceux-ci sont utilis\'es dans un objectif diff\'erent du nôtre. Le modèle que nous construisons a l'avantage d'être facilement expressif (peu de codage doit être employ\'e) et modulaire (on peut modifier une partie du modèle, comme le langage de programmation ou la syntaxe, avec un impact minime sur les autres parties).


\section{The realizability model}

We give a detailed construction of the realizability model: the logic used, the lambda-calculus and the realizability relation. We also give the standard theorem for a realizability model, namely the adequacy, which guaranties that we construct a consistant theory.


The model we construct is based on higher-order logic (HOL). This is a multi-sorted logic with a sort $\Prop$ denoting the set of propositions (hence the higher order qualification).
The choice of HOL as our syntax is mainly motivated by its expressiveness. In HOL, it is possible to easily talk about predicates on sets over a given sort $S$, by the sort $S \to \Prop$. %Using this logic and the basic sorts $\Nat/\Bool/\List$ makes it easy to express statements about trees (which are predicates on lists) such as  $\FT$ and $\WKL$.


In realizability models, universal quantifications are usually interpreted as intersections without computational content, for instance having truth values of first-order universal quantifications satisfying $|\forall x,\varphi(x)|=\bigcap_{n\in\bN}|\varphi(n)|$. In particular, a realizer in $|\forall x,\varphi(x)|$ should realize $\varphi(n)$ for any $n\in\bN$ \emph{without} knowing the value of $n$. Dually, realizers of existential statements witness the existence of a valid instantiation for the existentially quantified variable without actually computing its value.
However, we seek a computational interpretation of the Fan Theorem such that \emph{its realizer actually computes} the uniform bound for the bar.
To overcome this, we rely on a standard technique and use \emph{relativized quantifications}: instead of $\forall x,\varphi(x)$, we consider
formulas of the shape $\forall x,\sortPredS{\Nat}{x} \to \varphi(x)$, where the proposition $\sortPredS{\Nat}{x}$ states that $x$ is indeed a natural. A realizer of this statement will now have to realize  $\sortPredS{\Nat}{n}\to \varphi(n)$, that is to realize $\varphi(n)$ provided a realizer of $\sortPredS{\Nat}{n}$, \emph{i.e.} a term computing the value of $n$. The realizability model is introduced in \Cref{s:realizability}, we begin with the definition of HOL formulas for our interpretation.

In order for our realizers to compute with trees, that is with path (via functions in $\bN\to\bB$), prefixes of paths (\emph{i.e.} lists on booleans) and (decidable) sets of lists, we therefore consider sorts accounting for natural numbers, lists, booleans, propositions and functions, while formulas are usual HOL propositions extended with a relativization predicate for each sort.

% This last construction is used to underline the standard elements and to give them a computational meaning. For example, the difference between $\forall \varx^{\Nat},\varphi(\varx)$ and $\forall \varx^{\Nat},\sortPredS{\Nat}{\varx} \to \varphi(\varx)$ is that the second proposition can be proved by a computation depending on the integer $\varx$, meanwhile the first one must be proved by a proof of $\varphi$ uniform on $\varx$.

\begin{definition}[Sorts]
  The set of sorts $\Sort$ is inductively defined by:
  \[S,T \Coloneq \Nat \mid \Bool \mid \List(S)\mid S \times T \mid S \to T \mid \Prop\]
\end{definition}

% \begin{definition}[HOL terms]
%  HOL formulas are inductively defined by :
%   \[\begin{array}{rcl}
%   \termt,\termu &\Coloneq &\varx \mid  \termlam \varx.\termt \mid \termt(\termu) \mid  \langle \termt, \termu \rangle \mid \termpi_1(\termt)
%   \mid \termpi_2(\termt)\\
%   &\mid&
%   \termZ \mid S(\termt)
%   \mid \termrec_{\Nat}(\termt,\termu,\termv)\\
%   &\mid&\termtt
%   \mid \termff
%   \mid \termrec_{\Bool}(\termt,\termu,\termv) \\
%   &\mid& \termnil
%   \mid \termt\termcons\termu
%   \mid \termrec_{\List(S)}(\termt,\termu,\termv)
%   \end{array}\]
% \end{definition}


\begin{definition}[Formulas]
 Formulas are inductively defined by:
  $$\varphi,\psi \Coloneq S(\termt) \mid \varphi\to\psi \mid \varphi \land \psi \mid \forall x^S,\varphi \mid \exists x^S,\varphi \eqno(S\in\Sort)$$
where $\termt$ is a HOL-term (\textit{c.f.} \Cref{def:hol_terms})
\end{definition}

With this set of sorts, we give a syntax for well-typed terms. We use bold font for the syntactic HOL terms, to distinguish them from the $\lambda$-calculus terms we introduce later. The only exception is to denote an element in $\Prop$, for which we use $\varphi,\psi\ldots$

The HOL term constructors are akin to system T: there is a constructor for abstraction and application for functions, constructors for pairing and projections, canonical constructors and recursor for our data types ($\Nat,\Bool,\List$). The main difference is on the logical aspect, as we also add propositional constructors for the formula above.
% Those constructors are the conjunction $\land$, the implication $\to$, the two quantifications $\forall/\exists$ and the construction $\sortPred{\termt}$.
% \tlnote{Maybe ignore this definition, it's straightforward.}
% \begin{definition}[HOL-context]
  As usual, we fix a denumerable set $\Xt$ of variables which we denote by $\varx,\bvary,\ldots$ and we define \emph{(HOL-)contexts} as lists of pairs $(\varx : S)$ where $\varx \in \Xt$ and $S \in \Sort$. We denote the set of (HOL-)contexts by $\Hctx$ and will write its elements $\Gamma,\Delta,\ldots$
% \end{definition}


\begin{definition}[HOL-terms]
  \label{def:hol_terms}
  A well-typed (HOL-)term $\termt$ of sort $S$ in a context $\Gamma$, which we denote by $\Gamma\vdash \termt:S$, is defined by the inductive relation given in Figure 1.%\ref{fig:hol_terms}.
  \end{definition}
  \begin{figure}[t]
  \begin{center}
  \ruleUnary{$(\varx : S) \in \Gamma$}{$\Gamma\vdash \varx : S$}
  \quad
  \ruleUnary{$\Gamma, \varx : S \vdash \termt : T$}{$\Gamma\vdash \termlam \varx.\termt : S \to T$}

   \vspace{0.2cm}

  \ruleBinary{$\Gamma\vdash\termt : S \to T$}{$\Gamma\vdash\termu : S$}{$\Gamma\vdash \termt(\termu) : T$}
\quad
  \ruleBinary{$\Gamma \vdash \termt : S$}{$\Gamma\vdash \termu : T$}{$\Gamma \vdash \langle \termt, \termu \rangle : S \times T$}

   \vspace{0.2cm}
  \ruleUnary{$\Gamma\vdash \termt : S \times T$}{$\Gamma\vdash \termpi_1(\termt) : S$}
  \quad
  \ruleUnary{$\Gamma\vdash \termt : S \times T$}{$\Gamma\vdash \termpi_2(\termt) : T$}

  \vspace{0.2cm}
  \ruleAx{$\Gamma\vdash \termZ : \Nat$}
  \quad
  \ruleUnary{$\Gamma\vdash \termt : \Nat$}{$\Gamma\vdash \termS(\termt) : \Nat$}
  %\quad

  \vspace{0.2cm}
  \ruleTernary{$\Gamma\vdash \termt : S$}{$\Gamma\vdash \termu : \Nat \to S \to S$}{$\Gamma\vdash \termv : \Nat$}{$\Gamma\vdash \termrec_{\Nat}(\termt,\termu,\termv) : S$}

  \vspace{0.2cm}
  \ruleAx{$\Gamma\vdash \termtt : \Bool$}
  \quad
  \ruleAx{$\Gamma\vdash \termff : \Bool$}
  %\quad

  \vspace{0.2cm}
  \ruleTernary{$\Gamma\vdash \termt : S$}{$\Gamma\vdash \termu : S$}{$\Gamma\vdash \termv : \Bool$}{$\Gamma\vdash \termrec_{\Bool}(\termt,\termu,\termv) : S$}

  \vspace{0.2cm}
  \ruleAx{$\Gamma\vdash \termnil : \List(S)$}
  \quad
  \ruleBinary{$\Gamma\vdash \termt : S$}{$\Gamma\vdash \termu : \List(S)$}{$\Gamma\vdash \termt\termcons\termu : \List(S)$}
  %\quad
  \vspace{0.2cm}
  \ruleTernary{$\Gamma\vdash \termt : T$}{$\Gamma\vdash \termu : S \to \List(S) \to T \to T$}{$\Gamma\vdash \termv : \List(S)$}{$\Gamma\vdash \termrec_{\List(S)}(\termt,\termu,\termv) : T$}

  \vspace{0.2cm}
  \ruleBinary{$\Gamma\vdash \varphi : \Prop$}{$\Gamma\vdash \psi : \Prop$}{$\Gamma\vdash \varphi \to \psi : \Prop$}
  \quad
  \ruleUnary{$\Gamma, \varx : S \vdash \varphi : \Prop$}{$\Gamma\vdash \forall \varx^S, \varphi : \Prop$}

  \vspace{0.2cm}
  \ruleBinary{$\Gamma\vdash \varphi : \Prop$}{$\Gamma\vdash \psi : \Prop$}{$\Gamma\vdash \varphi \land \psi : \Prop$}
  \quad%\vspace{0.2cm}
  \ruleUnary{$\Gamma, \varx : S \vdash \varphi : \Prop$}{$\Gamma\vdash \exists \varx^S, \varphi : \Prop$}

  \vspace{0.2cm}
  \ruleUnary{$\Gamma\vdash : \termt : S$}{$\Gamma \vdash \sortPred{\termt} : \Prop$}
  \end{center}
   \label{fig:hol_terms}
   \caption{Well-typed HOL terms}   %\emnote{put in a figure*?}
  \end{figure}
% \end{definition}

A model of such a syntax must give a semantic interpretation of each sort. In particular, there must be an interpretation for the sort $\Prop$. From the constructors of this sort, the interpretation of $\Prop$ must have a function interpreting $\land$ and $\to$, and functions interpreting $\forall$ and $\exists$. We postpone the introduction of the exact interpretation of $\Prop$, and parametrize the model by a Heyting pre-algebra $H$ equipped with two functions $\bigcap,\bigcup : \powerset (H) \to H$ representing unions and intersections. Finally, to interpret $\Prop$, we need an interpretation of $\sortPred{\termt}$, for which we add an abstract encoding function $\cod -$ giving for each interpretation of a term an object of $H$.
The other sorts are given their natural set theoretic semantics.

\begin{definition}[Semantics]\label{def:sem}
  Let $(H,\leqH)$ be a Heyting pre-algebra with an intersection and a union operation $\bigcup,\bigcap : \powerset (H) \to H$.
  For each sort $S \in \Sort$, we associate a set $\sem S$:
  \begin{multicols}{2}
  \begin{itemize}
  \item $\sem{\Nat} \defeq \bN$
  \item $\sem{\Bool} \defeq \bB$
  \item $\sem{\List(S)} \defeq \sem S ^\star$
  \item $\sem{S \times T} \defeq \sem S \times_{\Set} \sem T$
  \item $\sem{S \to T} \defeq {\sem T}^{\sem S}$
  \item $\sem{\Prop} \defeq H$
  \end{itemize}
  \end{multicols}
  \noindent Let
  \[\begin{array}{ccccc}
  \cod - &:& \bigcup_{S \in \Sort} \sem{S} & \longrightarrow & H \\
  & & s &\longmapsto & \cod s
  \end{array}\]
  be a function associating to each object a set of codes (possibly empty).
  Each well-typed term $\Gamma \vdash \termt : S$ is mapped to its semantics
  \[
  \sem{\Gamma \vdash \termt : S} : \prod_{T \in \Gamma} \sem T \longrightarrow \sem S
  \]
  by the usual set-theoretic interpretation (the complete definition is given in \Cref{app:hol_semantic}). For example,
  \begin{multline*}
  \sem{\Gamma\vdash \termlam\varx.\termt : S \to T} \defeq\\
  (x \in \sem S) \mapsto \overrightarrow{x} \in \sem\Gamma) \mapsto \sem{\Gamma,\varx : S\vdash \termt : T}(\overrightarrow{x},x)
  \end{multline*}
  while we interpretation $\Gamma\vdash \sortPred \termt : \Prop$ as $\cod - \circ \sem{\Gamma\vdash \termt : S}$.

  For a given sort $S$, we will also write $\bS$ for $\sem S$.
\end{definition}

In all generality, the syntactic terms are given in a typing context, like $\Gamma\vdash \termt : S$. Thus, to construct an element of $\bS$, we need an environment $\rho$ which covers $\Gamma$.

\begin{notation}
  Given a HOL-context $\Gamma$ and a function $\displaystyle\termenv : \Xt \partialto \bigcup_{S \in \Sort} \bS$, we define
  \[\termenv\models \Gamma \defeq \forall (\varx : S) \in \Gamma, \termenv(\varx) \in \bS\]
  If $\Gamma \vdash \termt : S$ and $\termenv\models \Gamma$, then we write $\termenv(\termt)$ for the canonical element in $\bS$ associatied to this term.
\end{notation}


\subsection{Lambda-calculus}

The lambda-calculu we use is a variant of the usual lambda-calculus, enriched with
constructors and recursors for integers, booleans and lists of objects. We also add
pairs and a family of oracles.

\begin{definition}[Lambda-terms]
  We fix an enumerable set $\Xlam$ of $\lambda$-variables which will be written
  $x,y,\ldots$ and define inductively the set $\Lambda$ of lambda-terms by:
  \begin{align*}
    t,u  &\Coloneq  x \mid \lambda x.t \mid t\;u \\
    & \mid \langle t,u\rangle \mid \pi_1\;t \mid \pi_2\;t \\
    & \mid \lamZ \mid \lamS\;t \mid \rec_{\Nat}\;t\;u\;v\\
    & \mid \rtt \mid \rff \mid \rec_{\Bool}\;t\;u\;v\\
    & \mid \nil \mid t\append u \mid \rec_{\List}\;t\;u\;v\\
    & \mid \oracle_n
  \end{align*}
  where $n$ ranges over $\bN$.
\end{definition}

We assume the usual convention of $\alpha$-renaming of bound variables.

As the family $(\oracle_n)_{n \in \bN}$ represents oracles, the reduction will be parametrised by a list of functions $\bN \to \bB$.

\begin{definition}[Reduction context]
  Let $\funB$ be the set of set-functions $\bN \to \bB$. We denote by $\Oracle$ the set of lists of set-functions $\bN \to \bB$:
  \[\Oracle \defeq \List(\funB)\]
  Elements of this set will be called reduction contexts. For any $\enviro \in \Oracle$, the $i$-th function of the list will be written $\enviro[i]$ and the length of $\enviro$ will be written $|\enviro|$.
\end{definition}

We assume the usual convention for substitution without capture of free variables.

\begin{notation}
  Given a partial map $\rho : \Xlam \partialto \Lambda$ and a term $t$, the simultaneous substitution of $t$ by $\rho$ is denoted $\rho(t)$. If $\rho$ is the function defined on the only point $x \in \Xlam$ by $x \mapsto u$, then we will write $t[u/x]$.
\end{notation}

The reduction itself is the expected $\beta$-reduction, enriched by the reduction, for
any set-function $f : \bN \to \bB$ in the $i$-th place of the reduction context:
\[\oracle_i \;\encode{n} \mapsto \encode{f(n)}\]
where $\encode{n}$ means $\lamS^n\;\lamZ$, the canonical encoding of integers in this lambda-calculus, and the encoding of booleans is obviously $\rtt$ and $\rff$, corresponding respectively to $1 \in \bB$ and $0 \in \bB$.

\begin{definition}[Reduction]
  Let $\oracle \in \Oracle$.
  We define the relation $\mapsto_\sigma$ of immediate reduction by the following rules:
  \[\begin{array}{lclr}
    (\lambda x.t)u & \rediE & t[u/x] \\
    \pi_i \langle t_1,t_2\rangle & \rediE & t_i & \forall i \in \{1,2\} \\
    \rec_{\Nat}\;t\;u\;\lamZ & \rediE & t \\
    \rec_{\Nat}\;t\;u\;(\lamS\;v) & \rediE & u\;v\;(\rec_{\Nat}\;t\;u\;v) \\
    \rec_{\Bool}\;t\;u\;\rtt & \rediE & t \\
    \rec_{\Bool}\;t\;u\;\rff & \rediE &u \\
    \rec_{\List}\;t\;u\;\nil & \rediE &t \\
    \rec_{\List}\;t\;u\;(v \append w) & \rediE &u\;v\;w\;(\rec_{\List}\;t\;u\;w) \\
    \oracle_i\;\encode{n} & \rediE & \encode{\enviro[i](n)} & \forall i < |\enviro|
  \end{array}\]

  The reduction relation $\redE$ is the smallest relation containing $\rediE$ and which is compatible, meaning that it is closed by subterms. We write $\redERT$ for the reflexive and transitive closure of $\redE$.
\end{definition}

It is folklore to check that $\redE$ is confluent.

\begin{notation}
  For $\enviro,\enviroT \in \Oracle$, we write $\enviro \infOr \enviroT$ when $\enviro$ is a prefix of $\enviroT$. We also write the cylinder on $\enviro$ as follows:
  \[\cyl\enviro \defeq \{\enviroT \in \Oracle \mid \enviro \infOr \enviroT\}\]
\end{notation}

\begin{proposition}
  If $\enviro, \enviroT \in \Oracle$ are such that $\enviro \infOr \enviroT$, then $\redE \subseteq \red_{\enviroT}$.
\end{proposition}

\begin{proof}
  It suffices to show that $\rediE\subseteq \redi_{\enviroT}$, but this is straightforward by just comparing the list of redexes.
\end{proof}

The truth values of our predicates will be subsets of $\Lambda$. As is standard in realizability, thoses subsets need to be closed by anti-reduction.

\begin{definition}[Saturated set]
  Let $\enviro\in\Oracle$ and $A \subseteq \Lambda$.
  We call $A$ a $\enviro$-saturated set if it is closed by anti-reduction for $\redE$, and write $\SATE$ the set of saturated sets:
  \[A \in \SATE \defeq \forall t,u \in \Lambda, (\forall \enviroT \in \cyl{\enviro}, t \redP{\enviroT} u) \implies u \in A \implies t \in A\]
  We define the set of saturated subsets as follows:
  \[A \in \SAT \defeq \forall t,u \in \Lambda, (\forall \enviro \in \Oracle, t \redE u) \implies u \in A \implies t \in A\]
\end{definition}

\begin{remark}
  Using the fact that $\redE\subseteq\red_{\enviroT}$ for $\enviro\infOr\enviroT$, we know that if $t \redE u$ and $u \in A$, for $A \in \SATE$, then $t \in A$. Thus the condition of being saturated is a strengthening of the naive definition, which would be
  \[\forall t,u \in \Lambda, t \redE u \implies u \in A \implies t \in A\]
\end{remark}

The following result is folklore.

\begin{proposition}
  For any $\enviro \in \Oracle$, $\SATE$ is a complete sub-lattice of $\powerset(\Lambda)$. $\SAT$ is also a complete sub-lattice of $\powerset(\Lambda)$.
\end{proposition}

We now look at the relation between inclusion of reduction contexts and reductions.

\begin{proposition}
  For $\enviro\infOr\enviroT$, the following inclusion holds:
  \[\SATP{\enviroT}\subseteq \SATP{\enviro}\]
\end{proposition}

\begin{proof}
  Let $\enviro,\enviroT \in \Oracle$ such that $\enviro\infOr\enviroT$. Let $A \in \SATP{\enviroT}$, $t,u \in \Lambda$ such that $\forall \enviroTT \in \cyl{\enviro}, t \redP{\enviroTT} u$ and $u \in A$. Then, we know that for any $\enviroTT \in \cyl{\enviroT}, t \redP{\enviroTT} u$ as $\cyl{\enviroT}\subseteq \cyl{\enviro}$, so $t \in A$ as $A$ is $\enviroT$-saturated.
\end{proof}

We also give a lemma which simplfies some proofs of saturation.

\begin{lemma}\label{lem:inter_SAT}
  Let $(A_\enviro)_{\enviro\in\Oracle}$ be a family of sets such that $\forall \enviro\in\Oracle, A_\enviro \in \SATE$. Then $\displaystyle\bigcap_{\enviro \in \Oracle} A_\enviro \in \SAT$.
\end{lemma}

\begin{proof}
  Let $t,u \in \Lambda$ such that
  \[\forall \enviro \in \Oracle, t \redE u\]
  and $u \in \displaystyle\bigcap_{\enviro \in \Oracle} A_\enviro$. Then for any $\enviro \in \Oracle$, $u \in A_\enviro$ so (as $A_\enviro$ is saturated) $t \in A_\enviro$. So $t \in \displaystyle\bigcap_{\enviro\in\Oracle} A_\enviro$.
\end{proof}

The $\SATE$ (and $\SAT$) are also Heyting algebras.

\begin{proposition}
  Let $\sigma \in \Oracle$.
  Let $A,B \in \SATE$ (resp. $\SAT$). Then let
  \begin{align*}
    A \infSATE B &\defeq \exists t \in \Lambda, \forall u \in A, tu \in B\\
    A \impliesSATE B &\defeq \{t \in \Lambda \mid \forall u \in A, tu \in B\}\\
    A \landSATE B &\defeq \{t \in \Lambda\mid (\pi_1\;t \in A)\land (\pi_2\;t \in B)\}
  \end{align*}
  This defines a Heyting pre-algebra.
\end{proposition}

\begin{proof}
  We only prove the result for $\SATE$, the reasonning is the same for $\SAT$.
  
  The fact that the two operations are well defined is standard and uses the compatibility of the relation.
  
  Let's show that $\infSATE$ is the right adjoint to $\landSATE$. This means that for any $A,B,C \in \SATE$:
  \[A \infSATE B \impliesSATE C \iff A \landSATE B \infSATE C\]
  \begin{itemize}
  \item suppose there is $t$ such that for any $u \in A$, $tu \in B \impliesSATE C$. Thus, the function $(\lambda x. t\;(\pi_1\;x)\;(\pi_2\;x))$ is a witness that $A \landSATE B \infSATE C$
  \item conversely, if there is $t$ such that for any $u \in A \landSATE B$, $tu \in C$, then the function $\lambda x.\lambda y.t \langle x,y\rangle$ is a witness that $A \infSATE B \impliesSATE C$.
  \end{itemize}

  Hence $\SATE$ is a Heyting pre-algebra.
\end{proof}

Finally, we give the Heyting pre-algebra we will use for our construction. This one is given by in a similar way as for $\SAT$, but we require a Kripke-like condition for the implication.

\begin{definition}[$\SAT$ family Heyting pre-algebra]
  We define the following Heyting pre-algebra. For the carrier set, let
  \[\HeytingFamily \defeq \left\{\left. (A_\enviro)_{\enviro \in \Oracle} \in \prod_{\enviro \in \Oracle} \SATE \right\vert \forall \enviroT \in \Oracle, \enviro \infOr \enviroT \implies A_\enviro \subseteq A_{\enviroT}\right\}\]
  We define the pre-order as
  \[\varA \InfSATFam \varB \defeq \exists t \in \Lambda, \forall \enviro \in \Oracle, \forall u \in \termAE, t\;u \in \termBE\]
  the conjunction and the implication respectively as
  \begin{align*}
    \varA \AndSATFam \varB &\defeq \enviro \longmapsto \{t \in \Lambda \mid (\pi_1\;t\in \termAE)\land (\pi_2\;t \in \termBE)\} \\
    \qquad \varA \ImplSATFam \varB &\defeq \enviro \longmapsto \{ t \in \Lambda \mid \forall \enviroT \in \cyl{\enviro}, \forall u \in \termA{\enviroT}, t\;u \in \termB{\enviroT}\}
  \end{align*}
\end{definition}

\begin{proof}
  We show that this is indeed a Heyting pre-algebra with arbitrary meet and join. First, for the meet and join, the fact that each $\SATE$ is stable by arbitrary intersection and union means that we can define
  \[\bigcap_{i \in I} (A_{\enviro,i})_{\enviro \in \Oracle} \defeq \left(\bigcap_{i \in I} A_{\enviro,i}\right)_{\enviro \in \Oracle}\]

  We are left to show the adjunction between $\AndSATFam$ and $\ImplSATFam$. Let $\varA$, $\varB$ and $\varC$ be families, then stating that $(\varA \AndSATFam \varB) \InfSATFam \varC$ and $\varA \InfSATFam (\varB \ImplSATFam \varC)$ mean respectively:
  \begin{itemize}
  \item we find a term $t$ such that if $\pi_1 u \in \termAE$ and $\pi_2 u \in \termBE$ for any $\enviro \in \Oracle$, then $t\;u \in \termCE$ for any $\enviro \in \Oracle$
  \item we find a term $t$ such that, for any $\enviro,\enviroT \in \Oracle$ such that $\enviro\infOr \enviroT$, $u \in \termAE, v \in \termB{\enviroT}$, we have $t\;u\;v \in \termC{\enviroT}$
  \end{itemize}
  using the monotonicity of the families, the two terms can be converted one into the other respectively by the two following functions:
  \[t \longmapsto \lambda u.\lambda v.t\;\langle u,v\rangle \qquad
  t \longmapsto \lambda u. t\;(\pi_1\;u)\;(\pi_2\;u)\]

  Hence this gives us a Heyting pre-algebra with a meet and join operation.
\end{proof}


\subsection{Realizability relation}

The realizability relation relates the lambda-calculus and the HOL model. The first step to relate the two is to instantiate the semantic defined previously with a Heyting pre-algebra interpreting $\Prop$. Of course, this instantiation is done by taking $\sem{\Prop}$ to be the set of saturated sets. Be it $\SATE$ or $\SAT$, those sets are equiped with a Heyting pre-algebra structure and both intersection and union operation.

The only thing left to be defined is the $\cod -$ operation.

\begin{definition}[Code of an element]
  We define by induction, for any $S \in \Sort$ and any element $s \in \bS$, the set $\cod s \in \SAT$:
  \begin{itemize}
  \item if $S = \Nat$, then for $n \in \bN$,
    \[\cod{n} \defeq \enviro \mapsto \{ t \in \Lambda \mid t \redERT \overline n\}\]
  \item if $S = \Bool$, then
    \[\cod{\top} \defeq \enviro \mapsto \{ t \in \Lambda \mid t \redERT \rtt\}\quad\text{ and }\quad\cod{\bot} \defeq \enviro\mapsto \{ t \in \Lambda \mid t \redERT \rff\}\]
  \item if $S = \List(T)$, then
    \[\cod{\varepsilon} \defeq \enviro \mapsto\{ t \in \Lambda \mid t \redERT \nil \}\]
    and for all $x \in \bT^\star$, $y \in \bT$,
    \[\cod{x \smallfrown y} \defeq \enviro\mapsto \{ t \in \Lambda \mid \exists u \in \codE x, \exists v \in \codE y, t \redERT v \append u\}\]
  \item if $S = T \times U$, then for all $x \in \bT$, $y \in \bU$,
    \[\cod{(x,y)} \defeq \cod{x} \landSAT \cod{y}\]
    %\[\codE{(x,y)} \defeq \{ t \in \Lambda \mid (\pi_1\;t \in \codE x) \land (\pi_2\;t \in \codE y)\}\]
  \item if $S = T \to U$, then for all $f : \bT \to \bU$,
    \[\cod f \defeq \bigcap_{x \in \bT} \cod x \impliesSAT \cod{f(x)}\]
    %\[\codE f \defeq \{ t \in \Lambda \mid \forall s \in \bT, \forall \enviroT \supOr\enviro,\forall u \in \codP s \enviroT, tu \in \codP{f(s)}\enviroT\}\]
  \item if $S = \Prop$, then for all $\varA \in \SAT$,
    \[\cod{\varA} \defeq \varA\]
  \end{itemize}

  For each $\enviro\in\Oracle$, we define $\codE s$ by $\codE s \defeq (\cod s)_\enviro$.
\end{definition}

The fact that $\codE s \in\SATE$ is a straightforward induction on the set of sorts. We postpone the proof that $\enviro\infOr \enviroT \implies \codE s \subseteq \codP s {\enviroT}$ to the proposition \ref{prop.monotonicity}, as it is a particular case of a proposition on realizability.

We now have a semantic interpretation of our initial HOL syntax, where propositions are sets of terms. The realizability relation then defines, for a syntactic proposition $\vdash \varphi : \Prop$, an object $\sem{\varphi} \in \sem{\Prop}$. In a way, the relation is simply $t \in \cod{\rho(\varphi)}$ for $\Gamma\vdash \varphi : \Prop$ and $\rho\models \Gamma$, but we prefer to give an explicite definition of this relation.

\begin{definition}[Realizability relation]\label{def:real}
  Let $\enviro \in \Oracle$, $\Gamma \in \Hctx$, $\termenv \models \Gamma$ and $\Gamma\vdash \varphi : \Prop$. We define by induction the realizability relation:
  \begin{itemize}
  \item $t \realPP \sortPred{\termt} \defeq t \in \codE{\termenv(\termt)}$
  \item $t \realPP \varphi \to \psi \defeq \forall \enviroT \supOr\enviro, u \realP{\enviroT}{\termenv} \varphi \implies tu \realP{\enviroT}{\termenv}\psi$
  \item $t \realPP \varphi \land \psi \defeq (\pi_1\;t \realPP \varphi) \land (\pi_2\;t \realPP \psi)$
  \item $t \realPP \forall \varx^S, \varphi \defeq \forall s \in \bS, t\realP{\enviro}{\termenv[\varx \mapsto s]} \varphi$
  \item $t \realPP \exists \varx^S, \varphi \defeq \exists s \in \bS, t \realP{\enviro}{\termenv[\varx \mapsto s]} \varphi$
  \end{itemize}

  If $\termenv = \varnothing$ (in which case $\Gamma = \nil$), then we only write $t \realP{\enviro}{} \varphi$.

  We also define the universal realizability relation:
  \[t \realUP{\termenv} \varphi \defeq \forall \enviro \in \Oracle, t \realP{\enviro}{\termenv}\varphi\]
\end{definition}

The following proposition justifies that the definitions give saturated sets, in the sense that they are monotonous with regard to the order on reduction contexts.

\begin{proposition}\label{prop.monotonicity}
  let $\enviro, \enviroT \in \Oracle$, $\enviro \infOr \enviroT$, $\Gamma \in \Hctx, \termenv \models \Gamma, \Gamma \vdash \varphi : \Prop$. Then, for every $t$, we have the following implication:
  \[t \realPP \varphi \implies t \realP{\enviroT}{\termenv} \varphi\]
\end{proposition}

\begin{proof}
  We prove this by induction on $\varphi$ (and thus, for the case where $\varphi = \sortPred{\termt}$, by induction on $S$):
  \begin{itemize}
  \item if $S = \Nat$, then $t \realPP \sortPredS{\Nat}{\termt}$ means that $t \redERT \overline n$, thus $t \redRTP{\enviroT} \overline n$, so $t \realP{\enviroT}{\termenv} \sortPredS{\Nat}{\termt}$.
  \item if $S \in \{\Bool, \List(T), T \times U, \Prop\}$, the argument is the same.
  \item if $S = T \to U$, then this is just a consequence of the case of implication and universal quantification.
  \item if $\varphi = \psi \to \chi$, then let $\enviroTT \supOr\enviroT$ and $u \realP{\enviroTT}{\termenv} \psi$. By transitivity, $\enviro \infOr \enviroTT$, so by hypothesis on $t$, $tu \realP{\enviroTT}{\termenv} \psi$
  \item if $\varphi = \psi \land \chi$, then we can conclude directly by induction hypothesis
  \item the cases $\forall, \exists$ are the same: the induction hypothesis goes through without issue
  \end{itemize}
  Hence the monotonicity of realizability with regard to order on reduction contexts.
\end{proof}

\begin{notation}
  For a formula $\varphi$, we write
  \begin{align*}
    \realFP{\varphi}{\termenv}{} &\defeq \{ t \in \Lambda \mid t \realUP{\termenv} \varphi \}\\
    \realF{\varphi} &\defeq \{ t \in \Lambda \mid t \realU \varphi\}
  \end{align*}
%  \begin{align*}
%    \realFE{\varphi} &\defeq \{ t \in \Lambda \mid t \realPP \varphi \}\\
%    \realUFE{\varphi} &\defeq (\realFE{\varphi})_{\enviro \in \Oracle} \\
%  \end{align*}
\end{notation}

%\begin{proposition}
%  For every $\enviro \in \Oracle$, $\Gamma \in \Hctx$, $\termenv \models \Gamma$ and $\Gamma \vdash \varphi : \Prop$, $\realFE{\varphi} \in \SAT$ and $\realUFE{\varphi} \in \SAT$.
%\end{proposition}

%\begin{proof}
%  The proof is by induction on the realizability relation:
%  \begin{itemize}
%  \item if $\varphi = \sortPred{\termt}$, then we just use the fact that each $\codE{s}$ is in $\SAT$ (respectively that $\cod{s} \in \SAT$), so it's the case in particular for $\codE{\termenv(\termt)}$.
%  \item if $\varphi = \psi \to \chi$, then let $t,u \in \Lambda, t \red u$ and $u \realPP \psi \to \chi$. Let's prove that $t \realPP \psi \to \chi$. Let $\enviroT \supOr\enviro$ and $v \realP{\enviroT}{\termenv} \psi$, let's prove that $tv \realP{\enviroT}{\termenv} \chi$. By compatibility, $tv \red uv$, but as $u \realPP \psi \to \chi$, this means that $uv \realP{\enviroT}{\termenv} \chi$, hence (as $\realFP{\chi}{\termenv}{\enviroT}$ is saturated by induction hypothesis) $tv \realP{\enviroT}{\termenv} \chi$.
%  \item if $\varphi = \psi \land \chi$, then let $t,u \in \Lambda, t \red u$ and $u \realPP \psi \land \chi$. This means that both $\pi_1\;u \realPP \psi$ and $\pi_2\;u \realPP \chi$, but then $\pi_i\;t \red \pi_i\;u$ for $i \in \{1,2\}$, so by saturation of $\realFE{\psi}$ and $\realFE{\chi}$ (given by induction hypotheses), $\pi_1\;t \realPP \psi$ and $\pi_2\;t \realPP \chi$. Thus $t \realPP \psi \land \chi$.
%  \item if $\varphi = \forall \varx^S, \varphi$, then we use the fact that $\SAT$ is closed by intersection.
%  \item if $\varphi = \exists \varx^S, \varphi$, then we use the fact that $\SAT$ is closed by union.
%  \end{itemize}
%  Hence the result (the monotonicity for $\realUFE{\varphi}$ is already proved in proposition \ref{prop.monotonicity}).
%\end{proof}

\begin{notation}
  We define the relativised universal quantification of a proposition, $\forall \varx^{\{S\}}, \varphi$, as
  \[\forall \varx^{\{S\}}, \varphi \defeq \forall \varx^S, S(\varx) \to \varphi\]
  Thus, when realizing a formula like $\forall \varx^{\{S\}}, \varphi$, it suffices to suppose that there is an objet $s \in \bS$ with a code $t$ and t proves $\varphi$ using the data $t$.

  The relativised existantial quantification is defined by
  \[\exists \varx^{\{S\}}, \varphi \defeq \exists \varx^{S}, S(\varx) \land \varphi\]
  This type is naturally equiped with both
  \[\pi_1 \realU \varphi \land \psi \to \varphi \qquad \pi_2 \realU \varphi \land \psi \to \psi\]
  This means that, if $t\realP{\enviro}{\termenv} \exists\varx^{\{S\}}, \varphi(\varx)$, then there exists some $s \in \bS$ such that $\pi_1\;t$ is a code of $s$ (for the reduction context $\enviro$) and $\pi_2\;t \realP{\enviro}{\termenv[\bvary\mapsto s]} \varphi(\bvary)$ with $\bvary$ a fresh variable.
\end{notation}


\subsection{Adequacy}

To prove the adequacy, we first prove a substitution lemma.

\begin{lemma}
  Let $\enviro \in \Oracle$, $\Gamma \in \Hctx$, $\termenv \models \Gamma$ and the following typing derivations: $\Gamma, \varx : S \vdash \varphi : \Prop$, $\Gamma\vdash \termt : S$. The following equivalence holds:
  \[t \realPP \varphi[\termt/\varx] \iff t \realP{\enviro}{\termenv[\varx\mapsto \termenv(\termt)]}\varphi \]
\end{lemma}

\begin{proof}
  We must first prove by induction on terms $\termu$ that $\termenv(\termu[\termt/\varx]) = \termenv[\varx \mapsto \termenv(\termt)](\termu)$. Then, by induction on $\varphi$:
  \begin{itemize}
  \item if $\varphi = S(\termt)$, then we use the equality on terms to have $t \in \codE{\termenv(\termt)} \iff t \in \codE{\termenv(\termt)}$
  \item the other two cases are straightforward by using the induction hypothesis to replace $\realPP \varphi[\termt/\varx]$ by $\realP{\enviro}{\termenv[\varx\mapsto\termenv(\termt)]} \varphi$.
  \end{itemize}
  Hence the result.
\end{proof}

To postulate the adequacy, we need to give a typing system.


\begin{definition}[Typing system]
  We define a ($\lambda$-)context as a list $\Xi \in \List(\Xlam \times \Prop)$, whose elements we will write $(x : \varphi)$.

  The typing relation $\Gamma\mid \Xi \vdash t : \varphi$ is defined by induction by the rules:
  \begin{center}
    \AxiomC{$(x : \varphi) \in \Xi$}
    \RightLabel{Ax}
    \UnaryInfC{$\Gamma\mid \Xi \vdash x : \varphi$}
    \DisplayProof

    \vspace{0.2cm}
    \AxiomC{$\Gamma\mid \Xi, x : \varphi \vdash t : \psi$}
    \RightLabel{$\to_\mathrm i$}
    \UnaryInfC{$\Gamma\mid \Xi \vdash \lambda x.t : \varphi \to \psi$}
    \DisplayProof
    \quad
    \AxiomC{$\Gamma\mid \Xi \vdash t : \varphi \to \psi$}
    \AxiomC{$\Gamma\mid \Xi \vdash u : \varphi$}
    \RightLabel{$\to_\mathrm e$}
    \BinaryInfC{$\Gamma\mid\Xi\vdash t\;u : \psi$}
    \DisplayProof

    \vspace{0.2cm}
    \AxiomC{$\Gamma, \varx : S \mid \Xi \vdash t : \varphi$}
    \RightLabel{$\forall_\mathrm i$}
    \UnaryInfC{$\Gamma\mid \Xi \vdash t : \forall \varx^S, \varphi$}
    \DisplayProof
    \quad
    \AxiomC{$\Gamma\mid \Xi \vdash t : \forall \varx^S, \varphi$}
    \AxiomC{$\Gamma\vdash \termt : S$}
    \RightLabel{$\forall_\mathrm e$}
    \BinaryInfC{$\Gamma\mid \Xi \vdash t : \varphi[\termt / \varx]$}
    \DisplayProof
    
    \vspace{0.2cm}
    \AxiomC{}
    \RightLabel{$\Nat_\mathrm i^0$}
    \UnaryInfC{$\Gamma\mid Xi \vdash 0 : \Nat(\termZ)$}
    \DisplayProof
    \quad
    \AxiomC{$\Gamma\mid\Xi\vdash t : \Nat(\termt)$}
    \RightLabel{$\Nat_\mathrm i^S$}
    \UnaryInfC{$\Gamma\mid\Xi\vdash S\;t : \Nat(\termS(\termt))$}
    \DisplayProof

    \vspace{0.2cm}
    \AxiomC{}
    \RightLabel{$\Nat_\mathrm e$}
    \UnaryInfC{$\Gamma\mid \Xi\vdash \rec_{\Nat} : \forall X^{\Nat \to \Prop},X(\termZ) \to (\forall \varn^{\{\Nat\}}, X(\varn) \to X(\termS(\varn))) \to \forall \varn^{\{\Nat\}}, X(\varn)$}
    \DisplayProof

    \vspace{0.2cm}
    \AxiomC{}
    \RightLabel{$\Bool_\mathrm i^\top$}
    \UnaryInfC{$\Gamma\mid\Xi\vdash \rtt : \Bool(\termtt)$}
    \DisplayProof
    \quad
    \AxiomC{}
    \RightLabel{$\Bool_\mathrm i^\bot$}
    \UnaryInfC{$\Gamma\mid\Xi\vdash \rff : \Bool(\termff)$}
    \DisplayProof

    \vspace{0.2cm}
    \AxiomC{}
    \RightLabel{$\Bool_\mathrm e$}
    \UnaryInfC{$\Gamma\mid\Xi\vdash \rec_{\Bool} : \forall X^{\Bool \to \Prop}, X(\termtt) \to X(\termff) \to \forall \varx^{\{\Bool\}}, X(\varx)$}
    \DisplayProof
    
    \vspace{0.2cm}
    \AxiomC{}
    \RightLabel{$\List_\mathrm i^{\nil}$}
    \UnaryInfC{$\Gamma\mid\Xi\vdash \nil : \List(\termnil)$}
    \DisplayProof

    \vspace{0.2cm}
    \AxiomC{$\Gamma\mid\Xi\vdash t : S(\termt)$}
    \AxiomC{$\Gamma\mid \Xi\vdash u : \List(S)(\termu)$}
    \RightLabel{$\List_\mathrm i^{\append}$}
    \BinaryInfC{$\Gamma\mid\Xi\vdash t\append u : \List(S)(\termt \termcons \termu)$}
    \DisplayProof

    \vspace{0.2cm}
    \AxiomC{}
    \RightLabel{$\List_\mathrm e$}
    \UnaryInfC{$\Gamma\mid\Xi \vdash \rec_{\List} : \forall X^{\List(S) \to \Prop}, X(\termnil) \to (\forall \varx^{\{S\}}, \forall \vary^{\{\List(S)\}}, X(\vary) \to X(\varx \termcons \vary)) \to \forall \varx^{\{\List(S)\}}, X(\varx)$}
    \DisplayProof
  \end{center}
\end{definition}

\begin{theorem}[Adequacy]
  Let $\enviro \in \Oracle$, $\Gamma \in \Hctx$, $\termenv \models \Gamma$, formulas
  \[\begin{cases}
  \Gamma\vdash \varphi_i : \Prop \qquad \forall i = 1,\ldots,n\\
  \Gamma\vdash \varphi : \Prop
  \end{cases}\]
  and terms
  \[\begin{cases}
  t_i \realPP \varphi_i \qquad \forall i = 1,\ldots,n\\
  \Gamma \mid x_1 : \varphi_1,\ldots,x_n : \varphi_n \vdash t : \varphi
  \end{cases}\]
  then $t[t_i/x_i] \realPP \varphi$.
\end{theorem}

\begin{proof}
  The proof is by induction on the typing derivation:
  \begin{itemize}
  \item case Ax with $t = x_i$: the result is by hypothesis.
  \item case $\to_\mathrm i$ with $\lambda x.t : \psi \to \chi$: let $\enviroT \in \cyl{\enviro}$ and $u \realP{\enviroT}{\termenv} \psi$, let's prove that $(\lambda x.t)u \realP{\enviroT}{\termenv} \chi$. By hypothesis, $t[t_i/x_i][u/x] \realP{\enviroT}{\termenv} \chi$, but then it suffices to use the fact that $\realFP{\chi}{\termenv}{\enviroT}$ is closed by antireduction.
  \item case $\to_\mathrm e$ is straightforward by definition of realizing $\psi \to \chi$.
  \item case $\forall_\mathrm i$ with $t : \forall \varx^S, \psi$: by induction hypothesis, we know that for any $\termenv' \models \Gamma, \varx : S$, $t[t_i/x_i] \realP{\enviro}{\termenv'} \psi$. For any $s \in \bS, \termenv[\varx \mapsto s] \models \Gamma$, so $t[t_i/x_i] \realP{\enviro}{\termenv[\varx\mapsto s]} \psi$ which is, by definition, $t[t_i/x_i] \realPP \forall \varx^S,\psi$.
  \item case $\forall_\mathrm e$ with $t : \psi[\termt/\varx]$: by definition of realizing $\forall \varx^S, \psi$ and using the substitution lemma.
  \item we won't treat the cases $\Nat$ nor $\Bool$, as they are the same as the case $\List$.
  \item case $\List_\mathrm i^{\nil}$: indeed, $\nil \redERT \nil$.
  \item case $\List_\mathrm i^{\append}$: by induction hypothesis.
  \item case $\List_\mathrm e$: let $X$ be a predicate on $\List(S)$, $\enviroT \in \cyl{\enviro}$ and
    \[\begin{cases}
    u \realP{\enviroT}{\termenv} X(\termnil)\\
    v \realP{\enviroT}{\termenv} \forall \varx^{\{X\}}, \forall \vary^{\{\List(S)\}}, X(\vary) \to X(\varx \termcons \vary)\\
    w \realP{\enviroT}{\termenv} \List(S)(\varx)
    \end{cases}\]
    we want to prove that $t = \rec_{\List}\;u\;v\;w$ realizes $X(\varx)$.

    The object $\termenv(\varx)$ is a list. By induction, it is either $\nil$ or $s \append s'$ for two elements $s \in \bS, s' \in \bS^\star$:
    \begin{itemize}
    \item if $\termenv(\varx) = \nil$, then $w \redRTP{\enviroT} \nil$ and $\rec_{\List}\;u\;v\;w \redRTP{\enviroT} u$ so, indeed, $u \realP{\enviroT}{\termenv} X(\termnil)$
    \item if $\termenv(\varx) = s \append s'$, then $w \redRTP{\enviroT} w_1\append w_2$ for $w_1 \in \codP{s}{\enviroT}$ and $w_2\in\codP{s'}{\enviroT}$. In this case, by induction hypothesis, $\rec_{\List}\;u\;v\;w_2 \realP{\enviroT}{\termenv[\vary \mapsto s']} \List(S)(\vary)$, and
      \[\rec_{\List}\;u\;v\;w \redRTP{\enviroT} \rec_{\List}\;u\;v\;(w_1\append w_2) \redRTP{\enviroT} v\;w_2\;(\rec_{\List}\;u\;v\;w_2)\]
      so $\rec_{\List}\;u\;v\;w\realP{\enviroT}{\termenv} X(\varx)$.
    \end{itemize}
  \end{itemize}
  Hence the result, by induction.
\end{proof}

\begin{corollary}
  Let $\Gamma\in \Hctx, \termenv \models \Gamma$, formulas
  \[\begin{cases}
  \Gamma\vdash \varphi_i : \Prop \qquad \forall i = 1,\ldots,n\\
  \Gamma\vdash \varphi : \Prop
  \end{cases}\]
  and terms
  \[\begin{cases}
  t_i \realUP{\termenv} \varphi_i \qquad \forall i = 1, \ldots, n \\
  \Gamma\mid x_1 : \varphi_1,\ldots,x_n : \varphi_n \vdash t : \varphi
  \end{cases}\]
  then $t[t_i/x_i] \realUP{\termenv} \varphi$.
\end{corollary}

Thus, the sets
\[\ThReal \defeq \{ \varphi : \Prop \mid \exists t \in \Lambda, t \real \varphi\} \qquad \ThRealU \defeq \{ \varphi : \Prop \mid \exists t \in \Lambda, t \realU \varphi\}\]
are closed by logical consequence.


\section{The main result}
\label{s:realizability}

With our model constructed, and the proof that it describes a consistent theory done, we can describe the proof that this model satisfies $\FT$. We also show that $\WKL$ fails with a standard argument using Kleene tree.

\subsection{First exploration of the model}

Given our model, any closed term typable in system T can be written as a closed term and be realized by a term canonicaly associated to it. This means that, for example, the functions $+$ or $\times$ can be expressed both as objects of the sort $\Nat \to \Nat \to \Nat$ and as realizers computing these functions. We will give a list of functions written in system T, and which are thus uniformly realized.

\subsubsection{Boolean operations}

The usual connectives can be defined as boolean operations:
\begin{align*}
  \BoolNot &\defeq \rec_{\Bool}\;\rff\;\rtt &: \Bool \to \Bool\\
  \BoolOr &\defeq \lambda b.\rec_{\Bool}\;\rtt\;b &: \Bool \to \Bool \to \Bool\\
  \BoolAnd &\defeq \lambda b.\rec_{\Bool}\;b\;\rff &: \Bool \to \Bool \to \Bool\\
  \BoolImp &\defeq \lambda b.\rec_{\Bool}\;\rtt\;(\BoolNot\;b) &: \Bool \to \Bool \to \Bool\\
  \BoolEq &\defeq \rec_{\Bool}\;(\lambda x.x)\;\BoolNot &: \Bool \to \Bool \to \Bool
\end{align*}

These function extend to operations on lists of booleans:
\begin{align*}
  \bigwedge &\defeq \rec_{\List}\;\rtt\; (\lambda b\;\_\;b'. b\BoolAnd b')&: \List(\Bool) \to \Bool\\
  \bigvee &\defeq \rec_{\List}\;\rff\;(\lambda b\;\_\;b'. b\BoolOr b') &: \List(Bool) \to \Bool
\end{align*}

\subsubsection{Arithmetic}

The usual arithmetic functions are
\begin{align*}
  +&\defeq \lambda n.\rec_{\Nat}\;(\lambda\;\_.n)\;(\lambda\;\_\;m.\lamS\;m) &: \Nat \to \Nat \to \Nat\\
  \times&\defeq \lambda n.\rec_{\Nat}\;(\lambda\;\_.\lamZ)\;(\lambda\;\_\;m.m+n) &: \Nat \to \Nat \to \Nat\\
  n^m &\defeq \rec_{\Nat}\;(\lambda\;\_.\lamS\;\lamZ)\;(\lambda\;\_\;m.m\times n)\;m&: \Nat\\
  \predNat &\defeq \rec_{\Nat}\;\lamZ\;(\lambda\;n\;\_.n) &:\Nat \to \Nat\\
  - &\defeq \lambda n.\rec_{\Nat}\;n\;(\lambda\;\_\;m.\predNat\;m) &:\Nat\to\Nat\to\Nat
\end{align*}

we can also write the following function:
\begin{align*}
  ``\eqNat"& \defeq\rec_{\Nat}\;(\rec_{\Nat}\;\rtt\;(\lambda\;\_\;\_.\rff))\;(\lambda\;\_\;f. (\rec_{\Nat}\;\rff\;(\lambda\;m\;\_.f\;m))) &:\Nat \to \Nat \to \Bool\\
\end{align*}
which realizes the graph of the equality. Hence, it can be proved by double induction that
\[\forall n,m^{\{\Nat\}}, n = m \iff (n \eqNat m) = \rtt\]

We also define an enumeration of integer:
\begin{align*}
  \NatToList &\defeq \rec_{\Nat}\;\nil\;(\lambda\;n\;\ell.n\append\ell) &: \Nat \to \List(\Nat)
\end{align*}
which reduces on $\overline n$ to $[\overline k \mid k < n]$.

\subsubsection{Operations on lists}

We first define the list concatenation:
\begin{align*}
  \concat &\defeq \lambda \ell.\rec_{\List}\;\ell\;(\lambda a\;\_\;\ell'.a\append\ell') &: \List(S) \to \List(S) \to \List(S)
\end{align*}
Supposing that $S$ is a sort with decidable equality (meaning that there is a function $e : S \to S \to \Bool$ such that $s = s'$ is equivalent to $e(s,s') = \rtt$), then we construct the prefix order (which is decidable) by
\begin{multline*}
  \ell \ListInf \ell' \defeq \rec_{\List}\;(\lambda\_.\rtt)\;(\lambda a\;\_\;\ell_0.\rec_{\List}\;\rff\;(\lambda b\;\ell'_0\;\_.a =_S b \BoolAnd \ell_0\;\ell'_0)) \\:\List(S)\to\List(S)\to \Bool
\end{multline*}

Given a function $f : S \to T$, we can make the map function:
\[
\mapL \defeq \lambda f.\rec_{\List}\;\nil\;(\lambda a\;\_\;\ell.(f\;a)\append \ell) : (S \to T) \to \List(S) \to \List(T)
\]
and, for a function $f : S \to \Bool$, we can make the filter function:
\begin{multline*}
\filterL \defeq \lambda f.\rec_{\List}\;\nil\;(\lambda a\;\_\;\ell.\rec_{\Bool}\;(a\append\ell)\;\ell\;(f\;a)) \\: (S \to \Bool) \to \List(S) \to \List(S)
\end{multline*}

Given a function $f : \Nat \to \Nat$, we define the list of its $n$ first values as
\[\restr{f}{} \defeq \rec_{\Nat}\;\nil\;(\lambda\;n\;\ell.(f\;n) \append \ell)\]

For an integer $n$, we define it writting in base $2$ as follows:
\[
  \NatToBool \defeq \rec_{\Nat}\;\rff\;(\lambda\;\_\;\_.\rtt) :\Nat\to\Bool\]
\begin{multline*}
  f \defeq \lambda \langle \ell, n \rangle.\rec_{\Nat}\;\langle \ell,n\rangle (\lambda\;\_\;\_.\langle (\NatToBool(n\% 2)) \append \ell, n/2\rangle)\\
  : (\List(\Bool)\times\Nat) \to (\List(\Bool)\times\Nat)
\end{multline*}
\[
\IntBaseD n \defeq \rec_{\Nat}\;\langle\nil,n \rangle\;(\lambda\;\_.f)\;n : \Nat \to \List(\Bool)
\]

\subsubsection{Predicate given a bar}

Fix a monotoous predicate $B : \List(\Bool) \to \Prop$, \textit{i.e.} such that
\[\forall \ell, \ell'^{\List(\Bool)}, \ell \ListInf \ell' \implies B(\ell) \implies B(\ell')\]
a bar for this predicate is a realizer (in somme reduction context $\enviro$)
\[b \realP{\enviro}{} \forall \ell^{\{\List(\Bool)\}}, \exists n^{\{\Nat\}}, B(\restr \ell n)\]

Our goal is to construct a monotonous subset $C\subseteq B$ which is decidable, meaning that we must have a term $t : \List(\Bool) \to \Bool$ such that $C(\ell) \iff t\;\ell = \rtt$.

For this, the idea is to consider every list by the order given by our enumeration. For a given list $\ell$, the bar $b$ can be applied to the function $\ell 0^\infty$ to have a list $\ell'$ (either a prefix of $\ell$ or $\ell$ followed by a finite number of $0$).

The algorithm is the following to decide whether a list $\ell_0$ is in $C$. We consider every list $\ell$ of size $n \leq |\ell_0|$: $\ell_0 \in C$ exactly when there is some list $\ell'$ such that its associated $\ell'$ has as an extension $\ell_0$.

The idea is thus to add every extension of $\ell_0$ when the given extension is not already decided. By the fact that $B$ is monotonous and that the bar gives a predicate inside $B$, we know that $C$ is a subset of $B$. As $C$ is made by adding extensions, it is also monotonous, and because we only have to check a finite number of lists to decide whether a given list is in $C$, this is decidable.

The associated term is the following:
\begin{multline*}
C(b) \defeq \lambda \ell. \bigvee \mapL\;((\lambda \ell'. \ell'\ListInf \ell)\circ b \circ \ListEnum)\;(\NatToList\;(2^{|\ell| + 1} - 1)) \\: \List(Bool) \to \Bool
\end{multline*}

Then, we can define an algorithm which tries every list of a given size $n$ and checks if every such list is in $C$. If this converges, then the given $n$ it defines is a uniforme bar on $C$ (hence on $B$):

\begin{multline*}
  n_{C(b)} \defeq \Theta(\lambda f\;n.\rec_{\Bool}\;n\;(f\;(\lamS\;n))\\
  (\bigwedge\;\mapL\;(C\circ \ListEnum)\;(\restr{\NatToList\;(2^{n + 1} - 1)}{2^n})))\;\lamZ
\end{multline*}

where $\Theta$ is the Turing fixpoint, defined by
\[ \theta \defeq \lambda x\;y.y\;(x\;x\;y)\qquad
\Theta \defeq \theta\;\theta\]


\subsection{Weak Fan Theorem is realized}

Before proving that $\FT$ is realized, a continuity lemma is needed.

\begin{lemma}[Continuity]\label{lem:continuity}
  Let $\enviro\in\Oracle$ be a reduction context and $t \in \Lambda$ be a term such that
  \[t \realP{\enviro}{} (\Nat \to \Nat) \to \Nat\]
  Then for any $\alpha \realP{\enviro}{} \Nat \to \Nat$, there exists $n\in \bN$ such that
  \[\forall (\beta \realP{\enviro}{} \Nat \to \Nat), (\forall i < n, \beta\;\overline i \eqredE \alpha\;\overline i) \implies t\;\beta \eqredE t\;\alpha\]
\end{lemma}

\begin{proof}
  By hypothesis, we know that $t\;\alpha \redERT \overline n$ for some $n \in \bN$. By confluence, it is possible to chose any reduction strategy which is normalizing. In particular, we can chose the leftmost reduction strategy:
  \[t\;\alpha \stratETP l \overline n\]
  Without loss of generality, we assume $t$ to be of the form $\lambda x.u$ as $t$ will reduce to such a term, given it realizes a function and, applied to $\alpha$, returns a value.

  Thus, we are left to prove the result with $t[\alpha/x] \stratETP l \overline n$. By induction on the number of reduction steps, we prove that there exists a finite prefix such that for all $\beta$ with this prefix, $t[\beta/x] \stratETP l \overline n$. Let's proceed by case analysis on the first step reduction:
  \begin{itemize}
  \item the redex can be outside $\alpha$, meaning that there is $u$ such that $t[\alpha/x] \stratE l u[\alpha/x]$, then $t[\beta/x] \stratE l u[\beta/x]$ for any $\beta$. Using the induction hypothesis, the result follows.
  \item the redex can involve $\alpha$. In this case, because of the reduction strategy, we find a right context $E[\;]$ and a term $u$ such that
    \[t[\alpha/x] = E[\alpha\;u[\alpha/x]][\alpha/x]\]
    By regrouping reductions, we only focus on the inner term, namely $\alpha\;u[\alpha/x]$: the induction hypothesis will be applied to the resulting term because $\alpha\;u[\alpha/x]$ reduces at least once. Let $v \defeq u[\alpha/x]$.

    We claim that $v \redERT \overline i$ for some $i \in \bN$. Indeed, we can assume without loss of generality that $\alpha$ does not reduce on a term which does not realizes $\Nat$: it suffices to replace the term $\alpha$ with the term
    \[\rec_{\Nat}\;(\alpha\;0)\;(\lambda n\;x. \alpha(S\;n))\]
    to have a function which still realizes $\Nat \to \Nat$ but no longer reduces on terms which do not realize $\Nat$.

    Thus, $v \redERT \overline i$, which implies that $\alpha\;v \redERT \alpha \;\overline i$, but then $\alpha\;\overline i \redERT \overline{\alpha(i)}$ so, in the end,
    $\alpha\;v \redERT \overline{\alpha(i)}$ and (by confluence)
    \[\alpha\;v \stratETP l \overline{\alpha(i)}\]

    Now, by induction hypothesis, we find a prefix such that for any $\beta$ with this prefix, $E[\overline{\alpha(i)}][\beta/x] \stratETP l \overline n$ and $u[\beta/x] \stratETP l \overline i$. If needed, we can strengthen the condition by taking a longer prefix of alpha such that $\alpha(i)$ is contained in it. Then, for $\beta$ with this prefix:
    \begin{align*}
      t[\beta/x] &= E[\beta\;u[\beta/x]][\beta/x] \\
      &\redERT E[\beta\;\overline i][\beta/x] \\
      &\stratETP l E[\overline{\alpha(i)}][\beta/x] \\
      t[\beta/x] &\stratETP l \overline{n}
    \end{align*}
    hence the result.
  \end{itemize}

  Thus, by induction, we deduce that there exists such a prefix.
\end{proof}

Using this lemma and the $\lambda$-terms from the previous subsection, we can now realize $\FT$.

\begin{theorem}[Realization of Fan Theorem]\label{thm:FT}
  Using our previous definitions:
  \[\lambda b. n_{C_b}\;0 \realU \FT\]
\end{theorem}

\begin{proof}
  Let's prove that $\lambda b. n_{C_b}\;0 \realU\FT$. Let $\enviro\in\Oracle$ be a reduction context of length $m$ and
  \[b \realP{\enviro}{} \forall \alpha^{\{\Nat \to \Bool\}}, \exists n^{\{\Nat\}}, \restr \alpha n \in B\]
  for some function $B : \List(\bB) \to \SAT$ closed by extension. By saturation, it suffices to show that
  \[n_{C_b}\;0 \realP{\enviro}{} \exists n^{\{\Nat\}}, \forall \alpha^{\{\Nat \to \Bool\}}, \restr\alpha n \in B\]
  but we already know that $n_{C_b}\;0$ either reduces to an integer or diverges. If it reduces to an integer, then this integer is a uniform bound for $C$, but as $C \subseteq B$, it follows that $n_{C_b}\;0$ is a uniform bound for $B$.

  Now, suppose $n_{C_b}\;0$ diverges. This means that for any $n \in \bN$, there exists some word $w_n \in \bB^\star$ such that $w_n \notin C$. By applying Weak König's Lemma (in our meta-theory), we find an infinite path
  \[\alpha : \bN \to \bB\]
  such that for all $n \in \bN, \restr \alpha n \notin C$. We now enrich our reduction context with this alpha:
  \[\enviroT \defeq \enviro \smallfrown \alpha\]
  so that $\oracle_m \realP{\enviroT}{} (\Nat \to \Bool)(\alpha)$. By definition of $b$, we know that
  \[b\;\oracle_m\realP{\enviroT}{} \exists n^{\{\Nat\}}, \restr \alpha n \in B\]
  which we can destruct as some $n \in \bN$ such that
  \[\pi_1\;(b\;\oracle_m) \redRTP{\enviroT} n \qquad \pi_2\;(b\;\oracle_m)\realP{\enviroT}{} \restr \alpha n \in B\]
  Using lemma \ref{lem:continuity}, we can find an index $p\in \bN$ such that
  \[\forall (\beta \realP{\enviroT}{} \Nat \to \Bool), (\forall i < p, \beta\;\overline i \redRTP{\enviroT} \overline{\alpha(i)}) \implies \pi_1(b\;\beta) \redRTP{\enviroT} n\]

  Let
  \[M \defeq \max(n,p)\]
  by continuity, we know that $\pi_1\;(b\;\alpha_M) \redRTP{\enviroT}{} n$, so when computing $C_b\;\alpha_M$, there exists a list $\ell$ of length $\leq M$ such that $\restr\alpha{{\pi_1\;(b\;\ell 0^\infty)}}$ is a prefix of $\restr{\alpha}{M}$, so $\restr{\alpha}{M} \in C$. But by hypothesis, $\restr{\alpha}{M} \notin C$; hence a contradiction.

  This means that there is no case when $n_{C_b}$ diverges, so
  \[n_{C_b}\realP{\enviro}{} \exists n^{\{\Nat\}}, \forall \alpha^{\{\Nat \to \Bool\}}, \restr\alpha n \in B\]
  which means that $\lambda b.n_{C_b}\realU \FT$.
\end{proof}

Not only does the model satisfy $\FT$, but it also does not satisfy $\WKL$. By this, we mean that $\WKL\notin\ThRealU$, making our model a separating model between the two principles.

To prove that $\WKL$ is not realized, we use the Kleene tree construction (which can be relativized without issue). The tree, seen as a predicate over binary words, only contains those words which are different from any computable function in a finite time. For example, if $\ell$ is of length $n$, then we check whether $\ell_0$, its first value, is different from the value of $\Enum{}{0}(0)$ computed on $n$ steps (if it exists). This way, any finite path is computable, because a universal machine exists and can simulate the computation of a code on an input for a finite time. However, no infinite path can be found from a computable point of view: if $\alpha$ is an infinite path in $K$, this means that for any $\Enum{}{i}(i)\convcal$ (which means that it is computed in a finite time), we have that $\alpha_i \neq \Enum{}{i}(i)$, hence $\alpha$ is diagonaly non computable, hence non computable. In the realizability model, this implies that no path of $K$ is realized.

%\emnote{separation + eventuellement décrire ici l'arbre kleene}

\begin{theorem}
  There is no $t \in \Lambda$ such that $t \realU \WKL$.
\end{theorem}

\begin{proof}
  Suppose that there is $t \realU \WKL$. Let $\enviro\in\Oracle$, then $t \realP{}{\enviro} \WKL$. Let $\EnumE{}$ be an enumeration of $\sigma$-computable functions given by fixing a Gödel encoding of terms $t$. The Kleene tree $K_\enviro$ is defined as the set of lists $\ell \in \List(\Bool)$ such that for $i < |\ell|$, if $\EnumE{i}(i)\convcal[i]$ (meaning that the computation of $\EnumE{i}$ terminates in less than $i$ steps) then $\EnumE{i}(i) \neq \ell_i$.
  $K_\enviro$ is thus a tree with arbitrarily large finite sequences $\ell \in K$, but an finite path inside $K_\enviro$ cannot be $\enviro$-computable as it differs with every $\EnumE{i}$ on $i$. The realized functions $\Nat \to \Bool$ at context $\enviro$ being $\enviro$-computable, and taking a term $u$ encoding the premisses of $\WKL$ for $K_\enviro$ (the fact that $K_\enviro$ has arbitrarily large finite sequences), $t\;u$ should realize an infinite path in $K_\enviro$ at context $\enviro$, which is impossible.

  Hence $\WKL\notin\ThReal{\enviro}$, and $\WKL\notin\ThRealU$.
\end{proof}


\section{A robust interpretation of the Fan Theorem}

The realizability model developed in \Cref{s:realizability} relies on the ideas
of Lubarsky and Rathjen's work \cite{LubRat13} to prove $\FT$.
For the argument in this proof to work, the two main conditions that emerge are
\begin{itemize}
\item a notion of continuity for the realizers \emnote{of quantified relativization/individuals realized by a code}
\item having a Kripke-style notion of future worlds in which, given a tree $T$ at some world $\enviro$,
there exists a future $\enviroT$ in which a path in $T$ can be computed.
\end{itemize}
We shall now show the robustness of this approach: we shall highlight that the interpretation
is actually independent of the particular choice of a computational system as long as these conditions are satisfied.
To that end, we will use the notion of \emph{evidenced frames} introduced in \cite{CohMiqTat21}.
Evidenced frames (which we will shorten as EF) are combinatorial structures encompassing
the notion of a realizability model:
an EF contains a set of witnesses describing the realizers,
a set of formulas describing the set of truth values and
a relation akin to the logical inequality $\infSAT$, where the witness is made explicit.
In particular, it has the advantage of allowing one to abstract away
from the implementation details to show that any computational system satisfying
some specifications (\emph{e.g.}, memoization features) will induce a realizability model
satisfying the expected logical principles (\emph{e.g.}, countable choice~\cite{CohMiqTat21}).
In this section, we therefore seek for a general way to state the two conditions aboves
(as well as a few others which will arise later) using EFs.

In contrast with previous works using EFs~\cite{CohMiqTat21,GarMiq23,CohGruKirMiq25EffHOL,
CohGruKirMiq25mca}, where concrete structures used to define realizability interpretations
(\textit{e.g.} monadic combinatorial algebras) are proven to induce an EF,
here we will rather will focus only on EFs:
the concrete structure we study will be introduced as an additional structure
over EFs.

This section ends with the statement of the \Cref{thm:main}, which is a more abstract
variant of the \Cref{thm:FT}: in this version, the realizability structure realizing $\FT$ can have realizers of any form ($\lambda$-terms, Turing machines\ldots) as long as it gives an implementation of integers with its recursor, satisfies the two conditions mentioned above for the argument to work,
plus two additional technical conditions.

\subsection{Evidenced frames}

The realizability model we give can be defined by an evidenced frame.

\begin{definition}[Evidenced frame]
  An evidenced frame $\EFE$ is a pair of sets $(E,\Phi)$ equiped with:
  \begin{itemize}
  \item a ternary relation $\relEFdot \subseteq E \times \Phi \times \Phi$
  \item an element $\witID$ such that
    \[\forall \varphi \in \Phi, \relEF \witID \varphi \varphi\]
  \item there exists a function $\witCompDot : E \times E \to E$ such that
    \[\forall \varphi, \psi, \chi \in \Phi, \forall \witx, \wity \in E, \relEF\witx\varphi\psi \implies \relEF\wity\psi\chi \implies \relEF{\witComp\witx\wity}{\varphi}{\chi}\]
  \item an element $\PropTop \in \Phi$ such that there exists an element $\witTop \in E$ such that
    \[\forall \varphi \in \Phi, \relEF\witTop\varphi\PropTop\]
  \item a function $\PropAndDot : \Phi \times \Phi \to \Phi$ such that there exist a function $\witPairDot : E \times E \to E$ and two elements $\witFst,\witSnd \in E$ such that
    \[\begin{array}{l}
    \forall \varphi, \psi \in \Phi, \relEF\witFst{\PropAnd\varphi\psi}\varphi\\
    
    \forall \varphi, \psi \in \Phi, \relEF\witSnd{\PropAnd\varphi\psi}\varphi\\
    
    \forall \varphi, \psi, \chi \in \Phi,\forall \witx,\wity \in E, \relEF\witx\varphi\psi \implies \relEF\wity\varphi\chi\implies \relEF{\witPair\witx\wity}\varphi{\PropAnd\psi\chi}
    \end{array}\]
  \item a function $\PropImplDot : \Phi \times \powerset(\Phi) \to \Phi$ such that there exist a function $\witLam{} : E \to E$ and an element $\witEval$ such that
    \[\begin{array}{l}
    \forall \varphi,\psi \in \Phi, \forall \vec{\chi}\in \powerset(\Phi), \forall \witx \in E, (\forall \chi \in \vec \chi, \relEF\witx{\PropAnd\varphi\psi}\chi) \implies \relEF{\witLam\witx}\varphi{\PropImpl\psi{\vec{\chi}}}\\
    \forall \varphi \in \Phi, \forall \vec{\psi} \in \powerset(\Phi),\forall \psi \in \vec{\psi}, \relEF{\witEval}{\PropAnd{(\PropImpl{\varphi}{\vec\psi})}\varphi}\psi
    \end{array}\]
  \end{itemize}
\end{definition}

\begin{proposition}[Saturated sequences]
  Using our previous convention, we define the two sets
  \[E \defeq \Lambda \qquad \Phi \defeq \prod_{\enviro \in \Oracle} \SATE\]
  with the following elements:
  \begin{itemize}
  \item the relation is given by
    \[\relEF t {\varA}{\varB}\defeq \forall \enviro \in \Oracle, \forall u \in \termAE, tu \in \termBE\]
  \item the identity element is $\witID \defeq \lambda x.x$
  \item the composition function is
    \[\witComp t u \defeq \lambda x.u\;(t\;x)\]
  \item the truth proposition is $\PropTop \defeq \enviro \mapsto \Lambda$ and the evidence for this is $\witTop \defeq \lambda x.x$
  \item the conjunction function is $\PropAndDot \defeq (\varA,\varB) \mapsto \enviro \mapsto \termAE \landSATE \termBE$, the pairing function is
    \[\witPair t u \defeq \lambda x. \langle t\;x,u\;x \rangle\]
    with $\witFst \defeq \pi_1$ and $\witSnd \defeq \pi_2$
  \item the implication function is
    \[\PropImplDot \defeq (\varA,\FamB) \mapsto \enviro \mapsto \termAE \impliesSATE \bigcap_{\varB \in \FamB} \termBE \]
    with abstraction function $\witLam t \defeq \lambda x.\lambda y. t\;\langle x,y\rangle$ and evaluation element $\witEval \defeq \lambda x.\pi_1\;x\;(\pi_2\;x)$
  \end{itemize}
  this data gives an evidenced frame.
\end{proposition}

\begin{proof}
  We need to prove the properties of each element:
  \begin{itemize}
  \item for any $\varA \in \Phi$, $\relEF{\lambda x.x}{\varA}{\varA}$: for $\enviro \in \Oracle$, for any $t \in \termAE$, $(\lambda x.x)t \redE t$, so by saturation $(\lambda x.x)t \in \termAE$.
  \item for any $\varA,\varB,\varC\in \Phi$, $t,u \in \Lambda$ such that $\relEF t {\varA}{\varB}$ and $\relEF u {\varB}{\varC}$, we have $\relEF{\lambda x.u\;(t\;x)}{\varA}{\varC}$: for $\enviro \in \Oracle$, for any $v \in \termAE$, we know that $t\;v \in \termBE$ (by hypothesis on $t$) and thus that $u\;(t\;v) \in \termCE$ (by hypothesis on $u$), but $(\lambda x.u\;(t\;x))\;v \redE u\;(t\;v)$, hence the result by saturation.
  \item for any $\varA \in \Phi, \relEF{\lambda x.x}{\varA}{\PropTop}$ is automatic: any term instead of $\lambda x.x$ would satisfy this property.
  \item for any $\varA,\varB \in \Phi$, $\relEF{\pi_1}{\varA}{\varB}$ and $\relEF{\pi_2}{\varA}{\varB}$: for $\enviro \in \Oracle, t \in \PropAnd{\termAE}{\termBE}$, we know that $\pi_1\;t \in \termAE$ and $\pi_2\;t \in \termBE$, which is exactly the expected result.
  \item for any $\varA,\varB,\varC \in \Phi$, for any $t,u \in \Lambda$, if $\relEF t {\varA}{\varB}$ and $\relEF u {\varA}{\varC}$, then $\relEF{\lambda x.\langle t\;x,u\;x\rangle}{\varA}{\PropAnd{\varB}{\varC}}$: let $\enviro\in \Oracle, v \in \termAE$, then $t\;v \in \termBE$ and $u\;v \in \termCE$, so $\langle t\;v,u\;v \in \PropAnd{\termBE}{\termCE}$ (by the same argument that we used to say that the typing rule $\land_\mathrm i$ is adequate). As $(\lambda x.\langle t\;x,u\;x\rangle)\;v \redE \langle t\;v,u\;v\rangle$, the result follows by saturation.
  \item for any $\varA,\varB \in \Phi$, for any set $\FamC\in\powerset(\Phi)$ and evidences $t \in \Lambda$, if $\forall \varC \in \FamC, \relEF t{\PropAnd{\varA}{\varC}}{\varC}$ then $\relEF{\witLam t}{\varA}{\PropImpl{\varB}{\FamC}}$: let $\enviro \in \Oracle, u \in \termAE$, let's prove that $\witLam t\;u \in \PropImpl{\varB}{\FamC}$. Let $\varC \in \FamC$, it suffices to show that for any $v \in \termBE$, $\witLam t\;u\;v \in \termCE$, but as $\langle u,v\rangle \in (\PropAnd{\varA}{\varB})_\enviro$, we deduce that $t\;\langle u,v\rangle \in \termCE$. Moreover, $\witLam t\;u\;v \redE^2 t\;\langle u,v\rangle$, hence the result by saturation.
  \item for any $\varA \in \Phi, \FamB \in \powerset(\Phi)$ and $\varB \in \FamB$, $\relEF{\witEval}{\PropAnd{(\PropImpl{\varA}{\FamB})}{\varA}}{\varB}$: let $\enviro \in \Oracle$ and $t \in (\PropAnd{(\PropImpl{\varA}{\FamB})}{\varA})_\enviro$, then $\pi_1\;t \in (\PropImpl{\varA}{\FamB})_\enviro$ and $\pi_2\;t \in \termAE$, so for $\varB \in \FamB$, $\pi_1\;t\;(\pi_2\;t) \in \termBE$, hence the result by saturation as $\witEval\;t \redE \pi_1\;t\;(\pi_2\;t)$.
  \end{itemize}
\end{proof}




We will now generalize our proof of $\FT$ by abstracting its core over the
abstract framework provided by evidenced frames.
Scrutinizing the argument in the proof of \Cref{thm:FT},
besides the whole Kripke-like structure of our model,
we identify the following key ingredients:
\begin{enumerate}
\item the realizability model accounts at least for system T functions (for instance,
encodings allow one to talk about functions $\List(\Nat) \to \Bool$),
\item a function $f$ in the model can be called on a function $g$ defined later, $g$ can be non computable at the stage where $f$ is defined.
\item the functions $(\Nat \to \Nat) \to \Nat$ are continuous, in the sense that for any $f : (\Nat \to \Nat) \to \Nat$ and $\alpha : \Nat \to \Nat$ with a code, there exists a modulus $n$ such that $\forall \beta, \restr\alpha n = \restr \beta n \implies f(\alpha) = f(\beta)$.
\item realizers can do an unbounded search,
as long as the meta-theory can prove that the search will terminate.
\end{enumerate}

The Kripke-like semantic realizability relation we gave, and the fact we state a property
about a possible future, strongly leads us to a generalization on some Kripke model.
The category semantic way of giving a Kripke model is to consider a category $\bC$,
a pre-ordered set $(W, \leq)$ and to give a functor $F : W \to \bC$.
We will follow this guideline by constructing a well-designed category for our work,
where morphisms preserve the information we expect.


\subsubsection{Evidenced frames with data types}
Observe that the functions which the former conditions refer to
are actually individuals witnessed by codes in our model.
Technically, this was handled by means of HOL-terms of the appropriated sorts
and the relativization predicate $\sortPred{\termt}$.
Again, we would like to maintain the distinction between terms witnessing individuals (the \emph{data types})
and realizers for formulas.
Therefore, we extend the definition of evidenced frames to account for data types:
those are evidenced frames for which we explicitly add a representation of $\bN$.
This can be seen as taking an evidenced frame $\EFE$
and pulling back from its category of assemblies a natural number object.
For this definition, as we want at least system T functions,
we also require an encoding (which is the one we expect for assemblies)
of functions, and thus a minimal system T types syntax.


\begin{definition}[Evidenced Frame with data types]
  An evidenced frame with data types (EFdata, for short),
  is a tuple $(\Phi,E,\relEFdot, \cod \cdot, e_0, e_S, e_{\rec})$ where:
  \begin{itemize}
  \item $(\Phi,E,\relEFdot)$ is an evidenced frame.
  \item $\cod \cdot : \bN \to \Phi$ is an encoding function. We extend it naturally for types on
    \[T,U \Coloneq \bN \mid T \to U\]
    by, for any $f : T \to U$:
    \[\cod f \defeq \prod_{x \in T} \PropImpl{\cod x}{\cod{f(x)}}\]
  \item $\relEF{e_0}{\top}{\cod 0}$
  \item for all $n \in \bN$, $\relEF{e_S}{\cod n}{\cod{n + 1}}$
  \item for all functions $f : \bN \to \Phi$,
    \[\relEF{e_{\rec}}{\top}{\PropImpl{f(0)}{\PropImpl{\left(\prod_{n \in \bN}\PropImpl{\cod n}{\PropImpl{f(n)}{f(n+1)}}\right)}{\prod_{n \in \bN} \PropImpl{\cod n}{f(n)}}}}\]
  %\item for all type $T$ as previously defined, let $\rec_T$ be the recursor $T \to (\bN \to T \to T) \to \bN \to T$, then $\relEF{e_{\rec}}{\top}{\cod{\rec_T}}$
  \end{itemize}
\end{definition}

\begin{remark}
  For any $n \in \bN$, $\cod n$ is realized by $S^n 0$. Also, any system T definable function is realized by the naturally associated term.
\end{remark}

To construct the category of EFdatas, we also need to define morphisms.
The notion of morphism we introduce is only intended to state a correct generalized version
of \Cref{thm:FT}, they are (very) strict morphisms which preverve every construction
(they preserve the evidences, the functions on propositions, \textit{etc.})
and are not to be considered as the natural notion of morphism for EFdata.
For example, EF morphisms as defined in \cite{CohMiqTat21} do not require
the constructions and morphism to commute on the nose.

\begin{definition}[Strict EFdata morphism]\label{def:efdata}
  Let $\EFE_1,\EFE_2$ be two EFdata (each component of which will be written respectively $E_1,\Phi_1,\ldots$ and $E_2,\Phi_2,\ldots$). A morphism of EFdata is the data of two functions $F_{\Phi} : \Phi_1 \to \Phi_2$ and $F_E : E_1 \to E_2$ (we will just write $F$) commuting with every constructor of an EFdata.

  We define $\CatEFData$ as the category whose objects are EFdatas, and morphisms are those defined above.
\end{definition}

With this notion of morphism, we can formalize the \emph{future world} notion by taking
a pre-order $\bC$ and constructing a functor $\bC \to \CatEFData$.
This makes it easy to define the condition that in any world and infinite binary tree $T$
at this world, there is a future in which a path in $T$ exists.


To generalize our argument in the setting of EF, we need some computational conditions. To begin with, we need to
%\emnote{add a sentence on how this is a faithful generalization of relativized quantifications.}
%\emnote{to do}
have a strong notion underlying the quantifier $\exists$, that we call explicit existential. The explicit existential allows one to use a proof of $\exists x, \varphi$ to extract an object $a$ such that $\varphi$ is true on $a$. In the presence of effects, and even more so with classical logic,
the witness used to prove $\exists x, \varphi$ can be impossible to retrieve, making
this explicit existential not realized \cite{Herbelin05}.
The explicit existential is thus a natural condition: when taking the relativized version of $\exists$, we expect the proof to be able to give an explicit witness by its code, and explicit existential ensures that this code of witness indeed retrieves the witness.
\begin{definition}[Explicit existential]
  An EFdata is said to have explicit existential if there is an evidence $e$ such that for any system T type $\tau$ and any function $\varphi(-) : \tau \to \Prop$ there exists $a \in \tau$ such that
  \[\relEF{e}{\exists x^\tau, \cod x \land \varphi(x)}{\cod a \land \varphi(a)}\]
\end{definition}


\subsubsection{Continuity}
Now, we formalize the continuity condition in the setting of EFdatas.
Continuity in computable systems is well-studied notion, for which a wide literature exists especially
in constructive settings \cite{these_baillon}.
For the purpose of this work, we will only consider the following (weak) version of continuity.

\begin{definition}[Continuity]
  Let $(\Phi,E,\relEFdot, \cod\cdot, e_0,e_S,e_{\rec})$ be an EFdata. This EFdata is said to have continuity if
  \begin{multline*}
    \forall f : (\bN \to \bN) \to \bN, \forall \alpha : \bN \to \bN, \forall \relEF{t}{\top}{\cod f},
    \\\forall \relEF{u}{\top}{\cod \alpha}, \exists n \in \bN,
    \forall \beta:\bN\to\bN,\restr \alpha n = \restr \beta n \implies f(\alpha) = f(\beta)
  \end{multline*}
\end{definition}
\begin{example}
Instead of oracles $\oracle_n$ interpreting functions $\bN \to \bN$, we could add as an oracle a program $\oracle$ such that $\oracle\;t$ reduces to $\rtt$ or $\rff$ depending on whether $t$ (weakly) terminates or not, supposing $t$ is closed and does not contain $\oracle$. In this case, it is possible to realize the function deciding if a path $\alpha : \bN \to \bB$ is the constant path $0$ by reading each bit of $\alpha$ until its first $1$. This function is not continuous, so the realizability model we could build on this computational system does not have continuity. For most reasonnable computational systems not adding non computable higher order function and without \texttt{quote} instruction, continuity is satisfied.
\end{example}


\subsubsection{Unbounded search}
% By analysing in details the proof of theorem \ref{thm:FT}, in addition to the two main conditions, two technical conditions are needed.

%\emnote{todo: introductory sentence}
Another principle that is used in the theorem \ref{thm:FT} but need not be realized in any EF is the ability to conduct an unbounded search. To find the uniform bound, we use a fixpoint akin to the $\mu$ operator from \cite{Kleene1943}. This is a reasonnable assumption: for instance, every PCA implements unbounded search thanks to the fixpoint combinator $Y$.
%This construction is not realized in some typed settings. For example, we can easily construct a realizability model of Heyting Arithmetic using system T functions: those functions are total and contain recursive primitive functions, hence they can't be stable under $\mu$.


%These condition arise from two constructions quite natural but not always true for an arbitrary realizability model:
%\begin{itemize}
%\item to find the uniform bound we use a fixpoint creating an unbounded search. This operation is akin to the $\mu$ operator from \cite{Kleene1943}, and is not realized in some typed settings. For example, we can easily construct a realizability model of Heyting Arithmetic using system T functions: those functions are total and contain recursive primitive functions, hence they can't be stable under $\mu$.

%\end{itemize}

\begin{definition}[Unbounded search]
  An EFdata is said to have unbounded search if there exists $e$ such that, for any $f : \bN \to \bN$ having at least one zero,
  \[\relEF{e}{\cod f}{(\exists (n : \bN), \cod n \land f(n) = 0)}\]
\end{definition}
\begin{example}
Typed realizability settings based on the $\lambda$-calculus or any other computational system where all realizers have to be
terminating would forbid such unbounded search,
but extension to PCF or untyped settings naturally allow for such an operator.
\end{example}


\subsubsection{The robust theorem}
We can now state our generalized theorem about $\FT$. The conditions are the one introduced above, but they are not needed to be satisfied in every world. In the proof of \Cref{thm:FT}, there are two worlds: the one from which we get the bar, and the one in which we get the path avoiding $C(b)$. The continuity and explicit existential only need to be satisfied in the latter and, up to going to an even further future, can be checked only on a dense set of worlds. The unbounded search is used on the base world (the one containing our model), which is the only use of this hypothesis.

\begin{theorem}\label{thm:main}
  Let $\EFE$ be an EFdata with unbounded search. Let $(W,\leq,\bOne)$ be an ordered set with lower bound $\bOne$ and $F : W \to \CatEFData$ be a functor such that $F(\bOne) = \EFE$. If the following conditions are satisfied:
  \begin{itemize}
  \item there is a dense subset $W' \subseteq W$ such that for any $w \in W', F(w)$ has continuity and explicit existential.
  \item for any $w \in W$, function $B : \List(\bB) \to \bB$ with an infinite path
  and a code in $w$, there is an element $w' \geq w$,
  a code $e_\alpha$ of an infinite path $\alpha$ in $B$
  and a code $e \in E_{w'}$ such that for any $n \in \bN$,
  $\relEF{e}{\cod n}{\restr \alpha n \in B}$
  \end{itemize}
  then $\EFE \real \FT$.
\end{theorem}

\begin{proof}
  We apply the proof of theorem \ref{thm:FT} in the general setting. To make the proof easier to read, we consider the types $\Bool$ and $\List(\Bool)$, as they can be encoded inside integers and the usual manipulations can be simulated on the encodings.

  The proposition we want to prove is that there exists $\witx \in E$ such that for any predicate $B : \List(\Bool) \to \Phi$ closed by extension,
  \begin{multline*}
    \witx\real (\forall \alpha^{\{\Nat \to \Bool\}}, \exists n^{\{\Nat\}}, \restr \alpha n \in B) \implies\\
    \exists n^{\{\Nat\}}, \forall \alpha^{\{\Nat \to \Bool\}}, \restr \alpha n \in B
  \end{multline*}
  
  We assume there is $b$ witnessing the bar on $B$, and the definition of $C(b)$ from it is system T defined: this ensures that we can define the same function using the witnesses $\witZ,\witS,\witRec$. In the definition of $n_{C(b)}$, we can't define the fixpoint by itself, but we can define the function $C'_b$ which, for any $n$, computes whether every $\ell \in \List(\Bool)$ of size $n$ is inside $C(b)$. This function has a code $e_{C'_b}$.

  By hypothesis, the evidenced frame has unbounded search, so we can apply it to $\lnot \circ C'_b$ (because we search for a $n$ such that $C'_b(n) = 1$). Combining it with the code $e_{C'_b}$, this gives an evidenced of a uniform bound. Thus, all that is left to prove is that $\lnot \circ C'_b$ indeed has a $0$.

  Suppose that $C'_b(n) = 0$ for any $n$. This means that for any $n \in \bN$, there is a list of size $n$ avoiding $C$. Using Weak König's Lemma on $\complement C$ (which is a tree), we find a path $\alpha : \bN \to \bB$ such that $\restr \alpha n \notin C$ for any $n \in \bN$. By hypothesis on our functor $W \to \CatEFData$, we find $w' \in W$ and a code $e_\alpha \in \cod \alpha$ in $w'$. By the other hypothesis, we can assume (up to going to a future world) that $F(w')$ has continuity.

  Let $g$ be the morphism from $\EFE$ to $F(w')$. By construction of morphisms, we have
  \[g(b)\real g(\forall \alpha^{\{\Nat \to \Bool\}}, \exists n^{\{\Nat\}}, \restr \alpha n \in B)\]
  but as $g$ commutes with $\cod -$, with $\implies$, and with quantification $\forall$, we deduce that
  \[g(b)\real \forall \alpha^{\{\Nat \to \Bool\}}, \exists n^{\{\Nat\}}, \restr \alpha n \in B\]
  hence, $g(b) e_{\alpha} \real \exists n^{\{\Nat\}}, \restr \alpha n \in g(B)$. Using the fact that $g(b)$ is a code of a function $(\bN \to \bN) \to \bN$, we can apply continuity to find $m$ such that $\restr \alpha m = \restr \beta m \implies g(b) \alpha = g(b) \beta$ for any $\beta : \bN \to \bN$ with a code. This means that we can apply the same argument as in theorem \ref{thm:FT} to have a contradiction~: taking $M$ the maximum of this $m$ and of the encoded $n^{\{\Nat\}}$ by $g(b) e_{\alpha}$ (that we can take by explicit existential), we get $\restr \alpha M \in C(b)$, which is contradicting the fact that $\alpha$ always avoids $C(b)$.
\end{proof}


\cite{BreHer21}
\switch
{\bibliographystyle{plainurl}}
{\bibliographystyle{ACM-Reference-Format}}
{\bibliographystyle{plainurl}}
\bibliography{biblio}
\end{document}
